% -*- coding: UTF-8 -*-
% vim: autoindent expandtab tabstop=4 sw=4 sts=4 filetype=tex
% chktex-file 27 - disable warning about missing include files

\chapter{Vorgehen}
\label{chap:procedure}

\section{Arbeitsorganisation}
\label{sec:organization}

\subsection{Regelmässige Treffen}
\label{subsec:meetings}

Regelmässige Besprechungen mit dem Betreuer der Arbeit halfen die
gesteckten Ziele zu erreichen und Fehlentwicklungen zu vermeiden. Der
Betreuer unterstützte den Autor dabei mit Vorschlägen. Die Treffen
fanden mindestens alle zwei Wochen statt, sie wurden in Form eines
Protokolles festgehalten. Das Protokoll findet sich in
Abschnitt~\ref{chap:10_meeting_minutes} im Anhang dieser Arbeit
(\ref{chap:attachment}).

\section{Projektphasen}
\label{sec:project_schedule}

\subsection{Meilensteine}
\label{subsec:milestones}

Um bei der Arbeit ein möglichst strukturiertes Vorgehen einzuhalten, wurden folgende Projektphasen gewählt:
\begin{itemize}
    \item Start der Projektarbeit
    \item Erarbeitung und Festhalten der Anforderungen
    \item Erarbeitung der theoretischen Grundlagen
    \item Erstellung der abschliessenden Dokumentation
    \item Erstellung eines Prototypen
\end{itemize}

Die Phasen \textit{Erarbeitung der theoretischen Grundlagen}, \textit{Erstellung
der abschliessenden Dokumentation} sowie \textit{Erstellung eines
Prototypen} liefen parallel ab. Erkenntnisse einer Phase flossen jeweils in die
anderen Phasen ein.

\subsection{Zeitplan / Projektphasen}
\label{subsec:timeschedule}

\begin{figure}[H]
    \begin{ganttchart}[
        vgrid,
        x unit=0.5cm,
        bar/.append style={fill=bfhgrey!50},
    ]{1}{22}
        \gantttitle{2015}{15}
        \gantttitle{2016}{7} \ganttnewline{}
        \gantttitlelist{38,...,52,1,2,3,4,5,6,7}{1} \ganttnewline{} % chktex 11: Disable "you should use \ldots to achieve.."
        % \gantttitlelist{1,...,21}{1} \\ % chktex 11: Disable "you should use \ldots to achieve.."
        \ganttbar{Projektstart}{1}{1} \ganttnewline{}
        \ganttlinkedbar{Anforderungen}{2}{3} \ganttnewline{}
        \ganttbar{Erarbeitung Theorie}{2}{10} \ganttnewline{}
        \ganttbar{Erstellung Prototyp}{6}{7} \ganttnewline{}
        \ganttbar{Dokumentation}{1}{10} \ganttnewline{}
        \ganttlinkedbar{Korrekturen}{9}{13} \ganttnewline{}
        \ganttmilestone{Rohfassung Dokumentation}{13} \ganttnewline{}
        \ganttbar{Korrekturen}{14}{19} \ganttnewline{}
        \ganttbar{Vorbereitung Präsentation/Verteidigung}{20}{21} \ganttnewline{}
        \ganttlinkedbar{Präsentation/Verteidigung}{22}{22} \ganttnewline{}
        \ganttmilestone{Abgabe Dokumentation}{22}
    \end{ganttchart}
    \caption{Zeitplan; Der Titel stellt Jahreszahlen, der Untertitel
    Kalenderwochen dar}\label{fig:timeschedule}
\end{figure}

\subsubsection{Projektstart}
\label{subsubsec:kick_off}

In der ersten Phase wurden die Meilensteine der Arbeit identifiziert und
skizziert. Um Details der Aufgabe zu verstehen, wurde das notwendige
Vorwissen über Algorithmen zur globalen Beleuchtung erarbeitet. Weiter wurde
die Grundlage dieser Dokumentation erstellt.

\subsubsection{Anforderungen}
\label{ssubsec:requirements}

In dieser Phase wurde das Ziel dieser Projektarbeit festgelegt. Vom Ziel
ausgehend wurden die dazu erforderlichen Projektphasen festgelegt.

\subsubsection{Erarbeitung theoretische Grundlagen}
\label{ssubsec:theoretical_background}

In dieser Phase ging es darum, sich in die Materie einzulesen und sich
diese anzueignen. Das Lesen und Bearbeiten von Publikationen führte zu
vielen neuen Denkanstössen und immer wieder zu neuen Recherchen. Die
Erkenntnisse dieser Phase flossen so stetig in die Dokumentation ein,
ergänzten und beeinflussten diese.

\subsubsection{Dokumentation}
\label{ssubsec:documentation}

Die vorliegende Arbeit entspricht der Dokumentation. Sie wurde während
der gesamten Projektarbeit stetig erweitert und diente zur Reflexion von
fertiggestellten Teilen.

\section{Technologien}
\label{sec:technologies}

\subsection{Tools und Software}
\label{subsec:tools_software}

\noindent\emph{Dokumentation und Prototyp}
\begin{description}
    \item[\LaTeX] Eine Makro-Sammlung für das \TeX-System. Wurde zur
        Erstellung dieser Dokumentation eingesetzt. Diese Dokumentation
        wurde mittels \LaTeX{} geschrieben.
    \item[Make] Build-Automations-Werkzeug, wurde zur Erstellung dieses Dokumentes eingesetzt.
    \item[CMake] Build-Automations-Werkzeug, wurde zur Erstellung des
        Prototypen eingesetzt.
    \item[zotero] Ein freies, quelloffenes Literaturverwaltungsprogramm
        zum Sammeln, Verwalten und Zitieren unterschiedlicher Online-
        und Offline-Quellen~\parencite{wikipedia_foundation_zotero_2015}.
    \item[VIM] Vi IMproved. Ein freier, quelloffener Texteditor zur
        Textbearbeitung. Wurde zum Verfassen der Dokumentation sowie zur
        Entwicklung des Prototypen eingesetzt.
    \item[LLVM] Low Level Virtual Machine. Eine Compiler-Architektur zum
        Kompilieren von Applikationen. Wurde zur Kompilation des
        Prototypen eingesetzt.
    \item[clang] Ein C-Sprachen-Frontend für LLVM.\ Wurde zur Kompilation
        des Prototypen eingesetzt.
\end{description}


\noindent\emph{Arbeitsorganisation}
\begin{description}
    \item[Git] Freie Software zur verteilten Versionsverwaltung, wurde
        für die Entwicklung dieser Dokumentation sowie des Prototypen verwendet. Die
        Projektarbeit findet sich
        unter~\href{https://www.github.com/sosterwalder/mte7101-project1}{GitHub}\footnote{
            \href{https://www.github.com/sosterwalder/mte7101-project1}{https://www.github.com/sosterwalder/mte7101-project1}
        }.
    \item[GitHub] Eine freie Hosting-Platform für Git mit Weboberfläche.
\end{description}

\subsection{Standards und Richtlinien}
\label{subsec:standards_guidelines}

\subsubsection{Programmcode}
\label{ssubsec:standards_guidelines:code}

Der Programmcode des Prototypen, welcher in C++ geschrieben wurde, folgt
den offiziellen Richtlinien für C++ von Google~\footnote{
    \href{https://google.github.io/styleguide/cppguide.html}{
        https://google.github.io/styleguide/cppguide.html
    }
}.

\subsubsection{Pseudecode}
\label{ssubsec:standards_guidelines:psuedocode}

Da der Autor dieser Arbeit bedingt durch seine tägliche Arbeit mit der
Programmiersprache \textit{Python} relativ bewandert ist, wird daher diese als
Sprache zur Beschreibung von Pseudocode verwendet.  Dabei wird aber kein
Augenmerk auf die formale Korrektheit, geschweige denn der Lauffähigkeit
des Pseudocodes gelegt.

\subsubsection{Projekt-Struktur}
\label{ssubsec:standards_guidelines:project_structure}

Um die Übersicht zu wahren und den Verwaltungsaufwand minimal zu halten,
wurde eine entsprechende Projekt-Struktur gewählt. Diese ist in
Auflistung~\ref{lst:project_structure} ersichtlich.

% Change caption
\begin{listing}
	\VerbatimInput[label=Projekt-Struktur,frame=single,numbers=left,firstline=23,lastline=39]{../README.md}
	\caption{Projekt-Struktur.}\label{lst:project_structure}
\end{listing}
