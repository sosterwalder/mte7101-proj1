% -*- coding: UTF-8 -*-
% vim: autoindent expandtab tabstop=4 sw=4 sts=4 filetype=tex
% chktex-file 27 - disable warning about missing include files

\section{Shading models (Modelle zur Schattierung)}
\label{sec:shading}

* Lighting model at each vertex, e.g. Lambert
* Shading: Compute the color of interior points between vertices
* Example image of mesh (vertices)

\subsection{Flat shading --- per vertex lighting}
\label{subsec:flat_shading}

* Calculate normal for each vertice by averaging line-segment normals, e.g.
  (normal(V1V2) + normal(V3V4)) / 2 to determine per vertex color
  * Image of calculated normals: line-segment (given through mesh) and vertex
* One color per face determined by vertex' color

\subsection{Gouraud shading --- face interpolated lighting}
\label{subsec:gouraud_shading}

Bei Gouraud-Shading handelt es sich um ein Shading-Verfahren, welches die
Farbintensitätswerte der Eckpunkte von Oberflächen eines Meshes interpoloiert.

Dazu wird mittels einem Beleuchtungsmodell an jedem Eckpunkt einer Oberfläche
(Polygon) die Farbe berechnet, als Beispiel dienen hier die Eckpunkte $V_{1}$ ,
$V_{2}$, $V_{3}$ und $V_{4}$. Um einen Farbintensitätswert für einen Eckpunkt
$V_{1}$ zu berechnen, wird der Normalenvektor des Eckpunktes (vertex normal)
benötigt. Es handelt sich dabei um den Normalenvektor der Oberfläche an der
Position des Eckpunktes $V_{1}$.

Um diesen zu berechnen, schlägt Gouraud die Berechnung des Durchschnittswertes
der Oberflächennormalen zweier adjazenter Liniensegmente (im 2D-Raum) bzw.
aller adjazenter Dreiecke (im 3D-Raum) vor.

\todo[inline]{Image explaining average calculation}

Die Berechnung des Normalenvektors eines Eckpunktes via Durchschnittswert ist
üblicherweise eine genügend gute Näherung der Oberflächennormale der
eigentlichen Oberfläche. Die Präzision hängt dabei aber klar von der
Granularität des Modelles (mesh) ab.

\subsection{Phong shading --- normal interpolated lighting}
\label{subsec:phong_shading}
