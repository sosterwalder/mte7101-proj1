\subsection{Ray Tracing}
\label{subsec:ray_tracing}

Sofern nicht anders vermerkt, basiert der folgende Abschnitt
auf~\cite{glassner_introduction_1989}[S. 1 bis 77].

Bei dem heute als Ray Tracing bekannten Verfahren, handelt es sich um
eine verbesserte Version des unter~\ref{subsec:ray_casting} genannten
Ray Casting Verfahrens. Dieses wurde im Juni 1980
durch~\citeauthor{whitted_improved_1980} verbessert.

Grundsätzlich ist die Idee, dass jeder Bildpunkt der gewählten Region
Licht aus nur einer Richtung erhält --- derer der Lichtstrahlen
(\textit{light rays}), welche durch die gewählte Region und die
sichtbare Bildfläche gehen. Somit trägt jedes Photon, welches aus dieser
Richtung kommt, zum Farbwert bzw.\ der Farbintensität eines Bildpunktes
bei.

Die Lichtstrahlen werden in dieser Richtung verfolgt, um festzustellen
wie das Licht ``erzeugt'' wird:
\begin{itemize}
    \item{Der Lichtstrahl trifft auf nichts} --- die Strahlenverfolgung
        wird beendet, der Bildpunkt wird mit der Farbe des Hintergrundes
        eingefärbt.
    \item{Der Lichtstrahl trifft auf eine Lichtquelle} --- die
        Strahlenverfolgung wird beendet, der Bildpunkt wird mit der
        Farbe und Intensität der Lichtquelle eingefärbt.
    \item{Der Lichtstrahl trifft auf eine Oberfläche} --- der Prozess
        wird von diesem Punkt aus neu gestartet um festzustellen, wie
        das Licht dort ``erzeugt'' wurde.
\end{itemize}

Wie der letzte Punkt zeigt, handelt es sich also um ein rekurisves Verfahren
und wird daher teilweise auch rekurisves Ray Tracing genannt.

\citeauthor{glassner_introduction_1989} unterscheidet analog zu den
genannten Fällen zwischen folgenden Typen von Strahlen:
\begin{itemize}
    \item{\textit{Pixel- oder Augen-Strahlen}:} Strahlen, welche das Licht von
        einer Lichtquelle durch einen Bildpunkt direkt zu der sichtbaren
        Bildfläche (also dem Betrachter) transportieren.

        Die Strahlengleichung lautet:

        $r(t) = x_{0} + t \cdot (S - x_{0})$

        wobei $x_{0}$ der Ausgangspunkt (also der Betrachter), $t$ ein
        Skalierungsfaktor zwischen $[1, \infty]$ und $S$ der Schnittpunkt mit der
        Bildfläche ist.

    \item{\textit{Licht- oder Schatten-Strahlen}:} Strahlen, welche das Licht von
        einer Lichtquelle direkt zu einer Oberfläche eines Objektes
        transportieren.

        Die Strahlengleichung lautet:

        $r(t) = p_{0} + t \cdot (L_{i} - p_{0})$

        wobei $p_{0}$ der Ausgangspunkt (also ein Punkt auf einer
        Oberfläche eines Objektes), $t$ ein Skalierungsfaktor zwischen
        $[\epsilon, 1]$ und $L_{i}$ der Ort der $i$-ten Lichtquelle ist.
        Bei Epsilon handelt es sich um einen Faktor zur Steuerung der
        Präzision, welcher verhindert, dass ein Objekt lokal auf sich
        selbst Schatten wirft.

    \item{\textit{Reflektions-Strahlen}:} Strahlen, welche das an einem
        Objekt gespiegelte Licht transportieren.

        Die Strahlengleichung lautet:

        $r(t) = p_{0} + t \cdot F$

        wobei $p_{0}$ der Ausgangspunkt (also ein Punkt auf einer
        Oberfläche eines Objektes), $t$ ein Skalierungsfaktor zwischen
        $[\epsilon, 1]$ und $F$ der reflektierte Vektor ausgehend vom
        eingehenden Strahl in Abhängigkeit der Oberflächennormale ist:
        $F = (S - x_{0}) - 2((S - x_{0}) \cdot N)N$.  Bei Epsilon
        handelt es sich wiederum um einen Faktor zur Steuerung der
        Präzision, welcher verhindert, dass ein Objekt lokal in sich
        selbst gespiegelt wird.

    \item{\textit{Transparenz-Strahlen}:} Strahlen, welche Licht
        transportieren, dass durch ein Objekt hindurch geht.

        Die Strahlengleichung der Transparenz-Strahlen wird untenstehend
        genauer erläutert.

\end{itemize}~\cite{glassner_introduction_1989}[S. 10]

Zur Berechnung des emitierten Lichtes an einem gewissen Punkt eines Objektes
wird in einem ersten Schritt die Lichtintensität dieses Punktes
bestimmt.

In einem weiteren Schritt wird berechnet, wie die Oberfläche an diesem
Punkt Licht in eine spezifische Richtung weitergibt.  Dies geschieht
durch die Lichtintensität des Punktes sowie dessen physikalischen
Eigenschaften. 

Die oben genannten Strahlenarten \textit{Schatten, Reflektion und
    Transparenz} beschreiben die Prinzipien, wie Licht an einem Punkt
ankommt und sich von diesem entfernt.

Die Eigenschaften von Licht, welches direkt von einer Lichtquelle kommt
und dann weiter emitiert bzw.\ propagiert wird, werden durch
die~\textit{Schatten-Strahlen} bestimmt. Bei jedem Schnittpunkt eines
Lichtstrahles mit einem Objekt wird ein Schatten-Strahl in Richtung jeder
Lichtquelle der Szene ``geworfen''. Trifft der Schatten-Strahl die
Lichtquelle, wird das Licht zur Berechnung der Farbe und Lichtintensität
genutzt. Trifft der Schatten-Strahl keine Lichtquelle, so wird das Licht
nicht berücksichtigt. Schatten-Strahlen generieren keine weiteren
Strahlen.

Trifft ein Lichtstrahl auf eine Oberfläche wird dieses zu gewissen
Teilen~\textit{absorbiert}, \textit{reflektiert} und \textit{gebrochen}.
Die jeweiligen Anteile hängen dabei vom Medium der Oberfläche bzw.\ des
Objektes, der Frequenz des Lichtes sowie dem Winkel zwischen dem
Eingangswinkel des Lichtstrahles und der Oberflächennormale ab. Licht
kann je nach Medium auch gestreut werden.

Die Eigenschaften von Licht, welches auf ein Objekt trifft und dann an
diesem gespiegelt wird, werden durch die~\textit{Reflektions-Strahlen}
bestimmt. Handelt es sich dabei um eine perfekte spekuläre Reflektion,
so wird die eingehende Energie, ausgehend nur in eine Richtung
reflektiert.
\todo[inline]{Add image and formula for perfect specular reflection and
    perfect diffuse reflection}

Und schliesslich werden die Eigenschaften von Licht, welches durch ein
Objekt hindurch geht und vielleicht von diesem gebrochen wird, durch
die~\textit{Transparenz-Strahlen} beschrieben. Tritt Licht in ein neues
Medium ein (z.B.\ von Luft in Wasser), so wird das Licht gebrochen.
Dabei wird die Bahn des Lichtstrahles nach innen hin abgelenkt, wenn das
Licht in ein Medium mit höherer Dichte eintritt.
\todo[inline]{Add image and formula for refraction (perfect specular
    transmission), total internal reflection and transmission}

Ein möglicher Algorithmus, wie solch ein Verfahren umgesetzt werden kann,
findet sich in~\ref{fig:ray_tracing:high_level}.

\begin{lstlisting}[language=Python,caption={Eine abstrakte Umsetzung des
        Ray Tracings \protect\footnotemark.},
    label={fig:ray_tracing:high_level},captionpos=b,emph={ray_trace}]
def ray_trace(current_point, direction):
"""Traces light rays from current point in space in given direction.

:param current_point: current point in space
:type  current_point: three dimensional point object
:param direction:     the direction to trace
:type  direction:     three dimensional vector

:return:              the color for the given point
:rtype:               float
"""

    # Set color to currently set background color
    color = self.background_color

    # "pixels" is a list of all pixels of the image plane
    for pixel in pixels:

        # Returns the ray passing through the given
        # pixel from the eye
        ray = ray_at_pixel(pixel)

        # object_list is a list containing all the meshes contained in
        # the scene to render
        for object in object_list:
            p = intersect(ray, object)

            if object.is_reflective:
                reflection_vector = reflect(direction, object)
                reflected_color   = ray_trace(p, reflection_vector)
                color = color + object.coeff * reflected_color

            if object.is_refractive:
                refracted_vector = refract(direction, object)
                refracted_color  = ray_trace(p, refracted_vector)
                color = color + object.coeff * refracted_color

            for light in incoming_lights_at(p):
                if not is_shadow_ray(p, light.position):
                    color = color + calc_lighting(p, direction, light, object)
            end
        end
    end

    return color
end
\end{lstlisting}
\footnotetext{Algorithmus in Pseudocode
    gemäss~\cite{glassner_introduction_1989}[Seite 283]}
