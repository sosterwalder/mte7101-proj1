\subsection{Globale Beleuchtungsmodelle}
\label{subsec:global_illumination_models}

Sofern nicht anders vermerkt, basiert der folgende Abschnitt auf~\cite{foley_computer_1996}[S. 775ff]

Globale Beleuchtungsmodelle beschreiben die reflektierte (Licht-) Intensität
eines Punktes aufgrund direkter Lichteinstrahlung durch Lichtquellen sowie
durch alles Licht, welches diesen Punkt nach Reflektion von bzw. Durchdringen
der eigenen oder anderer Oberflächen erreicht.

Bei globalen Beleuchtungsmodellen unterscheidet man zwischen
blickwinkelabhängigen Algorithmen, wie etwa Ray Tracing, und zwischen
blickwinkelunabhängigen Algorithmen, wie etwa Photon Mapping.

Blickwinkelabhängige Algorithmen verwenden eine Diskretisierung der sichtbaren
Fläche bzw. Bildfläche um zu entscheiden, an welchen Punkten, in Blickrichtung
des Betrachters, die Beleuchtungsberechnung durchgeführt werden soll.
Blickwinkelunabhängige Algorithmen hingegen diskretisieren und verarbeiten die
Umgebung um genügend Informationen für die Beleuchtungsberechnung zu haben.
Dies erlaubt ihnen die Beleuchtungsberechnung an einem beliebigen Punkt aus
einer beliebigen Blickrichtung.

Beide Arten von Algorithmen haben jedoch Vor- und Nachteile. So sind
blickwinkelabhängige Algorithmen gut geeignet um Spiegelungen, basierend auf
der Blickrichtung des Betrachtes, zu berechnen, eignen sich aber weniger um
gleichbleibende diffuse Anteile über weiter Flächen eines Bildes zu berechnen.
Bei blickwinkelabhängigen Algorithmen verhält es sich genau umgekehrt.

\subsubsection{Renderinggleichung}
\label{ssubsec:rendering_equation}

Die unter~\ref{subsec:global_illumination_models} genannten Verfahren versuchen
auszudrücken, wie sich Licht von einem Punkt im Raum zu einem anderen bewegt.
Dabei beschreiben sie die Intensität des Lichtes, ausgehend vom ersten Punkt
zum zweiten Punkt. Zusätzlich wird die Intensität des Lichtes, ausgehend von
allen anderen Punkten, welche den ersten Punkt erreichen, und zum zweiten Punkt
emitiert werden, beschrieben.

James (Jim) Kajiya stellte 1986 die so genannte Renderinggleichung auf, welche
genau dieses Verhalten beschreibt~\cite{kajiya_rendering_1986}
und~\cite{foley_computer_1996}:
\begin{equation}
    I(x, x') = g(x, x')[\varepsilon(x, x') + \int\limits_{S}\rho(x, x', x'')I(x', x'')dx'']
\end{equation}
wobei gilt:

\begin{itemize}
    \item $x, x' \text{und } x''$: Punkte in der Umgebung
    \item $ I(x, x')$:            Lichtintensität von Punkt $x'$ nach Punkt $x$
    \item $ g(x, x')$:            Ein auf die Geometrie bezogener Term\\
                                  \hspace*{4mm} $0$:     \hspace*{6mm} $x$ und $x'$ verdecken sich\\
                                  \hspace*{4mm} $1/r^2$: \hspace*{1mm} $x$ und $x'$ sehen sich, wobei $r$ die Distanz zwischen $x$ und $x'$ ist
    \item $\epsilon(x, x')$:      Intensität des Lichtes, welches von $x'$ nach $x$ emitiert wird
    \item $\rho(x, x', x'')$:     Intensität des Lichtes, welches von $x''$
                                  durch die Oberfläche bei $x'$ nach $x$
                                  gestreut wird
    \item $\int\limits_{S}$:      Integral über die Vereinigung aller Flächen,
                                  daher $ S = \bigcup{S_{i}} $\\
                                      Dies bedeutet, dass die Punkte $x$, $x'$
                                      und $x''$ über alle Flächen aller Objekte
                                      der Szene ``streifen''.  Wobei es sich
                                      bei $S_{0}$ um eine zusätzliche Fläche
                                      handelt, welche als Hintergrund verwendet
                                      wird.  $S_{0}$ ist dabei eine Hemisphäre,
                                      welche die gesamte Szene umspannt.
\end{itemize}
