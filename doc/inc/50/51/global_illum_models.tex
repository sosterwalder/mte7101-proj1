% -*- coding: UTF-8 -*-
% vim: autoindent expandtab tabstop=4 sw=4 sts=4 filetype=tex
% vim: spelllang=de spell
% chktex-file 27 - disable warning about missing include files

\subsection{Globale Beleuchtungsmodelle}
\label{subsec:global_illumination_models}

Sofern im Text nicht anders vermerkt, basiert der folgende Abschnitt
auf~\cite[S. 775ff]{foley_computer_1996}.

Globale Beleuchtungsmodelle beschreiben die (Licht-) Intensität eines
Punktes. Die Intensität setzt sich aus direkter und indirekter
Einstrahlung des Lichtes zusammen.

Direktes Licht kommt unbeeinflusst aus einer Lichtquelle direkt zum
Betrachter bzw.~zu einem Bildpunkt. Indirektes Licht entsteht durch
Diffusion, Reflexion und Transmission, je nach Beschaffenheit der Körper
sowie deren Oberfläche.

Bei globalen Beleuchtungsmodellen unterscheidet man zwischen Blickwinkel
abhängigen, z.B.~Ray Tracing, und zwischen Blickwinkel
unabhängigen Algorithmen, z.B.~Photon Mapping.

Blickwinkel abhängige Algorithmen verwenden eine Diskretisierung der
sichtbaren Fläche bzw.\ Bildfläche (Zerlegung in kleine Abschnitte, also
Bildpunkte). Dadurch kann entscheiden werden, von welchen Punkten, in
Blickrichtung des Betrachters gesehen, die Berechnung der Beleuchtung
durchgeführt werden soll.

Blickwinkel unabhängige Algorithmen hingegen diskretisieren die
Umgebung. Sie stellen so genügend Informationen zur Verfügung, um die
Berechnung der Beleuchtung an einem beliebigen Punkt unabhängig von der
Blickrichtung des Betrachters zu berechnen.

Beide Arten von Algorithmen haben Vor- und Nachteile.  Blickwinkel
abhängige Algorithmen eignen sich gut für die Berechnung von
Spiegelungen, welche auf der Blickrichtung des Betrachters basieren.
Sie eignen sich aber weniger um gleichbleibende diffuse Anteile über
weite Flächen eines Bildes zu berechnen.\\
Bei Blickwinkel abhängigen Algorithmen ist genau das Gegenteil der Fall.

\subsubsection{Renderinggleichung}
\label{ssubsec:rendering_equation}

Die unter~\ref{subsec:global_illumination_models} genannten Verfahren
berechnen, wie sich Licht von einem Punkt $A$ zu einem anderen Punkt $B$
im Raum bewegt. Dabei beschreiben sie die Intensität des Lichtes von $A$
nach $B$. Zusätzlich werden alle Intensitäten von beliebigen Punkten im
Raum, die den Punkt $A$ erreichen und zu Punkt $B$ reflektiert werden,
mit einberechnet.

James Kajiya stellte 1986 die so genannte Renderinggleichung auf, welche
dieses Verhalten
beschreibt~\parencites{kajiya_rendering_1986}{foley_computer_1996}:

\begin{equation}
    I(x, x') = g(x, x')[\varepsilon(x, x') + \int\limits_{S}\rho(x, x', x'')I(x', x'')dx'']
\end{equation}

Wobei gilt:

\begin{itemize}
    \item $x, x' \text{und } x''$\\
        Punkte im Raum.
    \item $ I(x, x')$\\
        Intensität des Lichtes, die von Punkt $x'$ zu Punkt $x$ gelangt.
    \item $ g(x, x')$\\
        Ein auf die Geometrie bezogener Term\\
        \hspace*{4mm} $0$:     \hspace*{6mm} $x$ und $x'$ verdecken
                               sich.\\
        \hspace*{4mm} $1/r^2$: \hspace*{1mm} $x$ und $x'$ sehen sich,
                               wobei $r$ die Distanz zwischen $x$ und
                               $x'$ ist.
    \item $\varepsilon(x, x')$\\
        Intensität des Lichtes, welches von $x'$ nach $x$ emittiert
        wird.
    \item $\rho(x, x', x'')$\\
        Intensität des Lichtes, welches von $x''$
        durch die Oberfläche bei $x'$ nach $x$
        gestreut wird.
    \item $\int\limits_{S}$\\
        Integral über die Vereinigung aller Flächen, daher $ S =
        \bigcup{S_{i}} $.\\ Dies bedeutet, dass die Punkte $x$, $x'$ und
        $x''$ über alle Flächen aller Objekte der Szene ``streifen''.
        Wobei es sich bei $S_{0}$ um eine zusätzliche Fläche handelt,
        welche als Hintergrund verwendet wird.  $S_{0}$ ist dabei eine
        Hemisphäre, welche die gesamte Szene umspannt.
\end{itemize}
