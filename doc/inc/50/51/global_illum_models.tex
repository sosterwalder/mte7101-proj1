\subsection{Globale Beleuchtungsmodelle}
\label{subsec:global_illumination_models}

Sofern nicht anders vermerkt, basiert der folgende Abschnitt
auf~\cite[S. 775ff]{foley_computer_1996}.

Globale Beleuchtungsmodelle beschreiben die reflektierte (Licht-) Intensität
eines Punktes aufgrund direkter Einstrahlung des Lichtes durch Lichtquellen sowie
durch alles Licht, welches diesen Punkt nach Reflexion von bzw. Durchdringen
der eigenen oder anderer Oberflächen erreicht.

Bei globalen Beleuchtungsmodellen unterscheidet man zwischen
Blickwinkel abhängigen Algorithmen, wie etwa Ray Tracing, und zwischen
Blickwinkel unabhängigen Algorithmen, wie etwa Photon Mapping.

Blickwinkel abhängige Algorithmen verwenden eine Diskretisierung der sichtbaren
Fläche bzw.\ Bildfläche um zu entscheiden, an welchen Punkten, in Blickrichtung
des Betrachters, die Berechnung der Beleuchtung durchgeführt werden soll.

Blickwinkel unabhängige Algorithmen hingegen, nehmen eine
Diskretisierung vor und verarbeiten die Umgebung um genügend
Informationen für die Berechnung der Beleuchtung zu haben.  Dies erlaubt
ihnen die Berechnung der Beleuchtung an einem beliebigen Punkt aus einer
beliebigen Blickrichtung.

Beide Arten von Algorithmen haben Vor- und Nachteile. So sind
Blickwinkel abhängige Algorithmen gut geeignet um Spiegelungen,
basierend auf der Blickrichtung des Betrachters, zu berechnen, eignen
sich aber weniger um gleichbleibende diffuse Anteile über weite Flächen
eines Bildes zu berechnen.  Bei Blickwinkel abhängigen Algorithmen
verhält es sich genau umgekehrt.

\subsubsection{Renderinggleichung}
\label{ssubsec:rendering_equation}

Die unter~\ref{subsec:global_illumination_models} genannten Verfahren versuchen
auszudrücken, wie sich Licht von einem Punkt im Raum zu einem anderen bewegt.
Dabei beschreiben sie die Intensität des Lichtes, ausgehend vom ersten Punkt
zum zweiten Punkt. Zusätzlich wird die Intensität des Lichtes, ausgehend von
allen anderen Punkten, welche den ersten Punkt erreichen, und zum zweiten Punkt
emittiert werden, beschrieben.

James (Jim) Kajiya stellte 1986 die so genannte Renderinggleichung auf,
welche genau dieses Verhalten
beschreibt~\parencite{kajiya_rendering_1986}~\parencite{foley_computer_1996}:

\begin{equation}
    I(x, x') = g(x, x')[\varepsilon(x, x') + \int\limits_{S}\rho(x, x', x'')I(x', x'')dx'']
\end{equation}

Wobei gilt:

\begin{itemize}
    \item $x, x' \text{und } x''$\\
        Punkte in der Umgebung.
    \item $ I(x, x')$\\
        Intensität des Lichtes von Punkt $x'$ nach Punkt $x$.
    \item $ g(x, x')$\\
        Ein auf die Geometrie bezogener Term\\
        \hspace*{4mm} $0$:     \hspace*{6mm} $x$ und $x'$ verdecken
                               sich.\\
        \hspace*{4mm} $1/r^2$: \hspace*{1mm} $x$ und $x'$ sehen sich,
                               wobei $r$ die Distanz zwischen $x$ und
                               $x'$ ist.
    \item $\epsilon(x, x')$\\
        Intensität des Lichtes, welches von $x'$ nach $x$ emittiert
        wird.
    \item $\rho(x, x', x'')$\\
        Intensität des Lichtes, welches von $x''$
        durch die Oberfläche bei $x'$ nach $x$
        gestreut wird.
    \item $\int\limits_{S}$\\
        Integral über die Vereinigung aller Flächen, daher $ S =
        \bigcup{S_{i}} $.\\ Dies bedeutet, dass die Punkte $x$, $x'$ und
        $x''$ über alle Flächen aller Objekte der Szene ``streifen''.
        Wobei es sich bei $S_{0}$ um eine zusätzliche Fläche handelt,
        welche als Hintergrund verwendet wird.  $S_{0}$ ist dabei eine
        Hemisphäre, welche die gesamte Szene umspannt.
\end{itemize}
