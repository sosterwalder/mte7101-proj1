% -*- coding: UTF-8 -*-
% vim: autoindent expandtab tabstop=4 sw=4 sts=4 filetype=tex
% chktex-file 27 - disable warning about missing include files

\chapter{Theoretischer Hintergrund}
\label{chap:theoretical_background}

\todo[inline]{Provide introductio}

\section{Ray Tracing}
\label{sec:ray_tracing}

Bei Ray Tracing handelt es sich um ein globales Beleuchtungsmodell, welches 1980 von Turner Whitted vorgestellt wurde.
\todo[inline]{Describe ray tracing}

\subsection{Ray Casting}
\label{subsec:ray_casting}

\todo[inline]{Describe ray casting}

\section{Volume Ray Casting}
\label{sec:volume_ray_casting}

\todo[inline]{Describe volume ray casting}

\subsection{Raymarching}
\label{subsec:raymarching}

\todo[inline]{Describe raymarching}
