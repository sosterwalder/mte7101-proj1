% -*- coding: UTF-8 -*-
% vim: autoindent expandtab tabstop=4 sw=4 sts=4 filetype=tex
% chktex-file 27

\chapter*{Abstract}
\label{chap:abstract}

Das Fachgebiet der Computergrafik hat seit jeher das Bestreben eine
möglichst realistische Darstellung von Szenen und Modellen zu erzeugen.
Eine Darstellung also, welche möglichst nahe an der Realität respektive
der menschlichen Wahrnehmung liegt.

Im Laufe der Zeit entstanden verschiedene Ansätze um eine solche
Darstellung zu erreichen. Ein Teilgebiet davon sind Beleuchtungsmodelle,
welche die Beleuchtung einer Darstellung bzw.~einer Szene berechnen.

Ein relativ realistisch wirkendes Beleuchtungsmodell ist Ray Tracing (zu
Deutsch Strahlen-Verfolgung), welches auf den physikalischen Grundlagen
von Licht und Materialien von Oberflächen basiert.

Möchte man jedoch mit auf Strahlen-Verfolgung basierenden Verfahren eine
Darstellung in Echtzeit erreichen, so waren diese Verfahren lange Zeit
zu langsam. Im Rahmen der Weiterentwicklung der Computer und vor allem
durch die Weiterentwicklung der Grafikkarten (GPUs), ist Ray Tracing
jedoch wieder in den Fokus der Darstellung von Szenen in Echtzeit
gerückt.

Diese Projektarbeit stellt die (theoretischen) Grundlagen zur Erzeugung
und Darstellung von Szenen und Modellen sowie ein spezielles Ray Tracing
Verfahren zur Darstellung von Bildern in Echtzeit vor. Es handelt sich
dabei um \textit{Volume Ray Casting} bzw.~\textit{Sphere Tracing}.

Um die Hypothese zu stützen, dass die komplexen, realistischer wirkenden
Verfahren wieder in den Fokus der Darstellung von Szenen in Echtzeit
gerückt sind, wurde ein Prototyp entwickelt, der Sphere Tracing in
Echtzeit auf der Grafikkarte (GPU) umsetzt.

Es konnte gezeigt werden, dass mit dem vorgestellten Verfahren mittels
moderner Grafikkarten eine Darstellung von Szenen und Modellen in
Echtzeit erreicht werden kann.
