% -*- coding: UTF-8 -*-
% vim: autoindent expandtab tabstop=4 sw=4 sts=4 filetype=tex
% chktex-file 27

\chapter*{Abstract}
\label{chap:abstract}

Das Fachgebiet ``Computergrafik'' war schon immer bestrebt eine
möglichst realitätsnahe Darstellung von Szenen und Modellen zu erzeugen.
Eine Darstellung, welche möglichst nahe an der Realität respektive der
menschlichen Wahrnehmung liegt.

Im Laufe der Zeit entstanden verschiedene Ansätze um eine solche
Darstellung zu erreichen. Ein Teilgebiet davon sind Beleuchtungsmodelle,
welche die Beleuchtung einer Darstellung bzw.~einer Szene berechnen.

Ein relativ realistisch wirkendes Beleuchtungsmodell ist Ray Tracing (zu
Deutsch Strahlen-Verfolgung), welches auf den physikalischen Grundlagen
von Licht und Materialien von Oberflächen basiert.

Diese Verfahren waren lange Zeit zu langsam um damit eine Darstellung zu
erreichen. Durch die Weiterentwicklung der Computer, vor allem der
Grafikkarten (GPUs), ist Ray Tracing jedoch für die Darstellung von
Szenen in Echtzeit wieder interessant geworden.

Diese Projektarbeit stellt die (theoretischen) Grundlagen zur Erzeugung
und Darstellung von Szenen und Modellen sowie ein spezielles Ray Tracing
Verfahren zur Darstellung von Bildern in Echtzeit vor. Dabei handelt es
sich um \textit{Volume Ray Casting} bzw.~\textit{Sphere Tracing}.

Um die Hypothese zu stützen, dass die komplexen, realistischer wirkenden
Verfahren wieder in den Fokus der Darstellung von Szenen in Echtzeit
gerückt sind, wurde ein Prototyp entwickelt, der Sphere Tracing in
Echtzeit auf der Grafikkarte (GPU) umsetzt.

Mit dem vorgestellten Verfahren gelingt mittels moderner Grafikkarten
eine Darstellung von Szenen und Modellen in Echtzeit.
