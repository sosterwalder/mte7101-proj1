% -*- coding: UTF-8 -*-
% vim: autoindent expandtab tabstop=4 sw=4 sts=4 filetype=tex
% chktex-file 27 - disable warning about missing include files

% Versionenkontrolle :
% -----------------------------------------------

\chapter*{}
\label{chap:versionen}

\begin{versionhistory}
    \vhEntry{0.1}{25.09.2015}{SO}{Initiale Erstellung des Dokumentes}
    \vhEntry{0.2}{27.09.2015}{SO}{Entwickeln einer initialen Struktur,
        Aufbau der Dokumentation, Entwicklung Kapitel~\ref{chap:procedure}, Beschreibung von
        Ray Tracing in Kapitel~\ref{chap:theoretical_background}, Hinzufügen des
        Kapitels~\ref{chap:20_administrative}, Einführen von TODO-Notizen
    }
    \vhEntry{0.3}{29.09.2015}{SO}{Einführen von Meeting
        Minutes~\ref{chap:10_meeting_minutes}, Erweitern des
        Kapitels~\ref{chap:theoretical_background} um Belechtungsmodelle, Beschreiben von
        lokalen Beleuchtungsmodellen
    }
    \vhEntry{0.4}{04.10.2015}{SO}{Hinzufügen von Standards und
        Richtlinien~\ref{subsec:standards_guidelines}, Erweitern des
        Kapitels~\ref{chap:theoretical_background} um globale
        Belechtungsmodelle~\ref{subsec:global_illumination_models} sowie Ray
        Casting~\ref{subsec:ray_casting}, Entfernen der Schriftart `cmbright', Hinzufügen
        des Kapitels über (implizite) Oberflächen~\ref{sec:surfaces}
    }
    \vhEntry{0.5}{11.10.2015}{SO}{Neuordnung des
        Kapitels~\ref{chap:theoretical_background}: Hinzufügen des Kapitels über Ray
        Tracing~\ref{sec:ray_casting_tracing} sowie über Darstellung von impliziten
        Oberflächen~\ref{sec:description_implicit_surfaces}
    }
    \vhEntry{0.6}{14.10.2015}{SO}{Hinzufügen von TODO-Notizen, Anpassung der
        Textdarstellung in Formeln
    }
    \vhEntry{0.7}{16.10.2015}{SO}{Hinzufügen einer Illustration des
        Phong-Beleuchtungmodelles in Kapitel~\ref{subsec:local_illumination_models},
        Abarbeiten von TODO-Notizen in diversen Kapiteln, Erweitern des Kapitels über
        (implizite) Oberflächen~\ref{sec:surfaces}
    }
    \vhEntry{0.8}{17.10.2015}{SO}{Hinzufügen einer Illustration des Ray Tracing
        Algorithmus in Kapitel~\ref{subsec:global_illumination_models}
    }
    \vhEntry{0.9}{19.10.2015}{SO}{Beginn Kapitel über Rendering von impliziten
        Oberflächen~\ref{sec:rendering_implicit_surfaces}
    }
    \vhEntry{0.10}{21.10.2015}{SO}{Erweiterung Kapitel über Rendering von
        impliziten Oberflächen, Einführung Kapitel über die Umsetzung eines
        Prototypen~\ref{chap:prototype}
    }
    \vhEntry{0.11}{24.10.2015}{SO}{Umstrukturierung des Dokumentes, Anpassung
        der Dokumentvorlage, Nachführen der Versionshistorie, Ausführen der
        Meeting Minutes vom 18.10.2015, Erstellen der Vorlage für Meeting
        Minutes vom 25.10.2015
    }
    \vhEntry{0.12}{31.10.2015}{SO}{Nachführen der Versionshistorie, Ausführen der
        Meeting Minutes vom 25.10.2015, Erstellen der Vorlage für Meeting
        Minutes vom 02.11.2015, Erweiterung des
        Kapitels~\ref{chap:prototype}, Abarbeiten von TODO-Notizen
    }
    \vhEntry{0.13}{07.11.2015}{SO}{Ausführen der Meeting Minutes vom
        02.11.2015, Erweiterung des Kapitels~\ref{chap:prototype}
    }
    \vhEntry{0.14}{15.11.2015}{SO}{Erarbeiten des
        Kapitels~\ref{sec:rendering_implicit_surfaces_shadows},
        Erweiterung des Kapitels~\ref{chap:prototype} um weiche Schatten, 
        Erstellen und Hinzufügen von Bildmaterial zu Kapiteln~\ref{subsec:ray_marching}
        und~\ref{subsec:sphere_tracing}.
    }
    \vhEntry{0.15}{29.11.2015}{SO}{Komplette Überarbeitung des Bildmaterials zu
        Kapiteln~\ref{subsec:ray_marching} und~\ref{subsec:sphere_tracing}.
    }
    \vhEntry{0.16}{06.12.2015}{SO}{Nachführen von Meeting Minutes und der
        Versionierung. Überarbeiten des Zeitplanes, weiter Überarbeitung des
        Bildmaterials. Hinzufügen des Kapitels~\ref{sec:shading}, Hinzufügen
        einer Einleitung zu Kapitel~\ref{chap:theoretical_background}.
    }
    \vhEntry{0.17}{17.12.2015}{SO}{Einführen des Kapitels über Modelle
        zur Schattierung~\ref{sec:shading}, Fertigstellung der
        Beschreibung von Gouraud-Shading~\ref{subsec:gouraud_shading}.
    }
    \vhEntry{0.18}{22.12.2015}{SO}{Erweiterung des Kapitels über Modelle
        zur Schattierung~\ref{sec:shading} um diverse Illustrationen.
    }
    \vhEntry{0.19}{28.12.2015}{SO}{Erweiterung des Kapitels über Modelle
        zur Schattierung~\ref{sec:shading} um diverse Illustrationen,
        Hinzufügen eines (bildlichen) Vergleiches der verschiedenen
        Modelle zur Schattierung, Abarbeiten von TODO-Noitzen,
        Hinzufügen eines Beispieles zur parametrischen Darstellung einer
        Kugel in Kapitel~\ref{sec:surfaces}, Korrektur diverser
        Rechtschreibefehler in
        Kapitel~\ref{sec:description_implicit_surfaces}.
    }
    \vhEntry{0.20}{03.01.2016}{SO}{Erweiterung des
        Kapitels~\ref{subsec:global_illumination_models} um Ray Casting
        und Ray Tracing Verfahren, Erstellen und Hinzufügen diverser
        Illustrationen zu Kapitel~\ref{sec:illumination_models},
        Erweiterung des Kapitels~\ref{sec:surfaces} um
        Abschnitt~\ref{ssubsec:distance_fields} über Distanzfelder,
        Überarbeitung der \LaTeX-Vorlage.
    }
    \vhEntry{0.21}{04.01.2016}{SO}{Aufteilen des
        Kapitels~\nameref{sec:illumination_models} in Unterkapitel,
        Erweiterung der
        Kapitel~\ref{subsec:ray_casting}~\nameref{subsec:ray_casting}
        und~\ref{subsec:ray_tracing}~\nameref{subsec:ray_tracing}.
    }
    \vhEntry{0.22}{05.01.2016}{SO}{Fertigstellung des
        Kapitels~\ref{subsec:ray_tracing}~\nameref{subsec:ray_tracing}.
    }
    \vhEntry{0.23}{07.01.2016}{SO}{Erstellen und Hinzufügen von diversen
        Illustrationen zu
        Kapitel~\ref{subsec:ray_tracing}~\nameref{subsec:ray_tracing},
        Hinzufügen eines Abschnittes über Shader in
        Kapitel~\ref{sec:shading}~\nameref{sec:shading}, Korrektur des
        Beispiels zur parametrischen Darstellung einer Kugel in
        Kapitel~\ref{sec:surfaces}, Abarbeiten von TODO-Notizen,
        Hinzufügen einer Einleitung zu
        Kapitel~\ref{sec:rendering_implicit_surfaces}~\nameref{sec:rendering_implicit_surfaces},
        Hinzufügen von Kommentaren zum Quellcode des Prototyps.
    }
    \vhEntry{0.24}{09.01.2016}{SO}{Einleitung der
        Projektarbeit~\ref{chap:10_introduction} überarbeitet, Korrektur
        der Beschreibung und Referenz von impliziten Oberflächen in
        Kpaitel~\ref{subsec:implicit_surfaces}, Erarbeiten des
        Fazits~\ref{chap:discussion_and_conclusion}, Beschreibung der
        Aufgabenstellung~\ref{chap:scope}
    }
    \vhEntry{0.25}{10.01.2016}{SO}{Fertigstellung der
        Kapitel~\ref{chap:scope}~\nameref{chap:scope}
        und~\ref{chap:procedure}~\nameref{chap:procedure}, Abarbeiten
        und Nachführen von verbleibenden TODO-Noitzen, Nachführen der
        Versionshistorie
    }
    \vhEntry{1.0}{11.01.2016}{SO}{Überarbeitung der Dokumentation bis
        und mit~\autoref{chap:theoretical_background}, Nachführen der
        Versionshistorie
    }
    \vhEntry{1.1}{13.01.2016}{SO}{Überarbeitung der Dokumentation
        ab~\autoref{chap:theoretical_background}, Nachführen der
        Versionshistorie
    }
\end{versionhistory}
