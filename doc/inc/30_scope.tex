% -*- coding: UTF-8 -*-
% vim: autoindent expandtab tabstop=4 sw=4 sts=4 filetype=tex
% vim: spelllang=de spell
% chktex-file 27 - disable warning about missing include files

\chapter{Aufgabenstellung}
\label{chap:scope}

\section{Motivation}
\label{sec:motivation}

Seit der Entstehung der Computergrafik Ende der 1960-er Jahre möchte man
möglichst realitätsnahe, echte Bilder generieren können. Angefangen mit
\textit{Photon Tracing} (1968), über \textit{Ray Tracing} (1980), bis hin
zu \textit{Photon Mapping} (1997/2001), wurden immer realistischer
wirkende Verfahren zur Darstellung von Bildern entwickelt.

Da der Aufwand möglichst realistisch wirkende Bilder zu erzeugen jedoch
relativ hoch ist, wurden diese Verfahren in der Regel nur für statische
Bilder verwendet. Und auch dort war und ist der Grad der Details
begrenzt bzw.\ an Rechenzeit gebunden. Möchte man mehr Details, so wird
mehr Rechenzeit benötigt. Daher wurde oder wird für eine Darstellung in
Echtzeit --- also eine Anzeige von mindestens 25 Bildern pro
Sekunde --- lange nur eine Näherung, basierend auf vereinfachten
\textit{Beleuchtungsmodellen} genutzt.

Durch die Weiterentwicklung der Computer und vor allem durch die
Weiterentwicklung der Grafikkarten (GPUs), sind die komplexeren,
realistischer wirkenden Verfahren, wie z.B.\ Ray Tracing, jedoch
wieder in den Fokus der Darstellung von Szenen in Echtzeit gerückt.

Diese Projektarbeit beschreibt theoretische Grundlagen zur Erzeugung von
möglichst realistisch wirkenden Bildern
anhand~\textit{Beleuchtungsmodellen}, \textit{praktischen Verfahren}
(wie z.B.\ Ray Marching oder Ray Tracing), \textit{Modellen zur
    Schattierung} und stellt schliesslich eine effiziente Methode des
Ray Tracings, das so genannte~\textit{Sphere Tracing} vor.

Um die Hypothese zu stützen, dass die komplexen, realistischer wirkenden
Verfahren wieder in den Fokus der Darstellung von Szenen in Echtzeit
gerückt sind, wurde ein Prototyp entwickelt, der Sphere Tracing in
Echtzeit auf der Grafikkarte (GPU) umsetzt.

\subsection{Demoszene}
\label{subsec:demoscene}

In den 1980er-Jahren entwickelte sich ``unter Anhängern der
Computerszene \dots{} während der Blütezeit der
8-Bit-Systeme''~\parencite{wikipedia_foundation_demoszene_2015} eine Bewegung
namens Demoszene. ``Ihre Mitglieder, die häufig Demoszener oder einfach
Szener genannt werden, erzeugen mit Computerprogrammen auf Rechnern so
genannte Demos – Digitale Kunst, meist in Form von musikalisch
unterlegten
Echtzeit-Animationen.''~\parencite{wikipedia_foundation_demoszene_2015}

Es handelt sich bei der Demoszene um ein sehr aktives und kreatives
Umfeld, in welchem die Technologie ständig an die Grenzen ihrer
Möglichkeiten gebracht wird. In diesem Umfeld entstehen regelmässig neue
Ideen zur Erzeugung von noch realistischer wirkenden Bildern. Dabei
findet eine wechselseitige Beeinflussung zwischen dem akademischen
Umfeld und der Demoszene statt.

So wurde auch das hier vorgestellte \textit{Sphere Tracing} Verfahren
relativ früh von in der Demoszene aktiven Personen aufgegriffen und
behandelt.

\section{Ziele und Abgrenzung}
\label{sec:objectives}

Diese Projektarbeit besteht aus drei Teilen. Der Beschreibung der
theoretischen Grundlagen, der Beschreibung des als Sphere Tracing
bekannten Verfahrens sowie der Umsetzung eines Prototypen.

In dieser Projektarbeit liegt der Fokus vor allem auf der
(mathematischen) Beschreibung und Anwendung von Oberflächen,
sowie der Darstellung von Bildern mittels Sphere Tracing.

Bei dem entwickelten Prototypen handelt es sich um eine Art
Machbarkeitsstudie. Diese zeigt, dass Bilder mittels Sphere Tracing in
Echtzeit erzeugt und dargestellt werden können.

Diese Projektarbeit dient als Vorarbeit und Grundlage für das MSE-Modul
\textit{MTE7102} --- ``Vertiefungsprojekt 2'' (Folgemodul).

\subsection{Vorgängige Aktivitäten}
\label{subsec:preliminaries}

Die Grundlagen dieser Projektarbeit waren dem Autor durch sein
vorgängiges Studium an der Berner Fachhochschule mit
Schwerpunkt~\textit{Computer Perception and Virtual Reality} bereits
bekannt, jedoch nur partiell.

Durch sein Interesse für die Demoszene wurde der Autor auf die Thematik
dieser Arbeit aufmerksam. Zuvor hatte er sich nie eingehend damit
beschäftigt.

\subsection{Neue Lerninhalte}
\label{subsec:new_learning_contents}

Zusätzlich zu den formalen Lerninhalten hatte die Arbeit für den Autor
die folgenden neuen Lerninhalte:
\begin{itemize}
    \item{Implizite Oberflächen}
    \item{Distanzfunktionen}
    \item{Distanzfelder}
    \item{Ray Marching}
    \item{Sphere Tracing}
    \item{Darstellung von (weichen) Schatten}
    \item{Praktische Umsetzung der vorgestellten Verfahren}
\end{itemize}
