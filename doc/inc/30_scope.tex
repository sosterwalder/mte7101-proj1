% -*- coding: UTF-8 -*-
% vim: autoindent expandtab tabstop=4 sw=4 sts=4 filetype=tex
% chktex-file 27 - disable warning about missing include files

\chapter{Aufgabenstellung}
\label{chap:scope}

\section{Motivation}
\label{sec:motivation}

Seit der Entstehung der Computergrafik Ende der 1960-er Jahre möchte man
möglichst realitätsnahe, echte Bilder generieren können. Angefangen mit
\textit{Photon Tracing} 1968, über \textit{Ray Tracing} 1980, bis hin
zum \textit{Photon Mapping} 1997/2001, haben sich immer realistischer
wirkende Verfahren zur Darstellung von Bildern entwickelt.

Da der Aufwand möglichst realistisch wirkende Bilder zu erzeugen jedoch
relativ hoch ist, wurden diese Verfahren in der Regel nur für statische
Bilder verwendet. Und auch dort war bzw.\ ist der Grad der Details
begrenzt bzw.\ an Rechenzeit gebunden. Möchte man mehr Details, so wird
mehr Rechenzeit benötigt. Daher wurde oder wird für eine Darstellung in
Echtzeit --- also dynamische Bilder mit mindestens 25 Bildern pro
Sekunde --- lange nur eine Näherung, basierend auf vereinfachten
\textit{Beleuchtungsmodellen} genutzt.

Durch die Weiterentwicklung der Computer und vor allem durch die
Weiterentwicklung der Grafikkarten (GPUs), sind die komplexeren,
realistischer wirkenden Verfahren, wie z.B.\ Ray Tracing jedoch
wieder in den Fokus der Darstellung von Szenen in Echtzeit gerückt.

Diese Projektarbeit beschreibt die theoretischen Grundlagen
anhand~\textit{Beleuchtungsmodellen}, Verfahren für eine möglichst
realistische Erzeugung von Bildern, \textit{Modellen zur Schattierung}
und stellt schliesslich eine effiziente Methode des \textit{Ray
    Tracings}, das so genannte~\textit{Sphere Tracing} vor.

Um die Hypothese zu stützen, dass die komplexen, realistischer wirkenden
Verfahren wieder in den Fokus der Darstellung von Szenen in Echtzeit
gerückt sind, wurde ein Protoyp entwickelt, der Sphere Tracing in
Echtzeit auf der Grafikkarte (GPU) umsetzt.

\subsection{Demoszene}
\label{subsec:demoscene}

In den 1980er-Jahren entwickelte sich ``unter Anhängern der
Computerszene in den \dots{} während der Blütezeit der
8-Bit-Systeme''~\cite{wikipedia_foundation_demoszene_2015} eine Bewegung
namens Demoszene. ``Ihre Mitglieder, die häufig Demoszener oder einfach
Szener genannt werden, erzeugen mit Computerprogrammen auf Rechnern so
genannte Demos – Digitale Kunst, meist in Form von musikalisch
unterlegten
Echtzeit-Animationen.''~\cite{wikipedia_foundation_demoszene_2015}

Es handelt sich bei der Demoszene um ein sehr aktives Umfeld, in dem die
Technologie ständig an die Grenzen des technisch machbaren gelangt. In
diesem Umfeld entstehen regelmässig neue Ideen zur Erzeugung von noch
realistischer wirkenden Bildern. Dabei findet eine wechselseitige
Beeinflussung zwischen dem akademischen Umfeld und der Demoszene statt.

\section{Ziele und Abgrenzung}
\label{sec:objectives}

Diese Projektarbeit besteht aus drei Teilen. Der Beschreibung der
theoretischen Grundlagen, der Beschreibung des als Sphere Tracing
bekannten Verfahrens sowie der Umsetzung eines Prototypen.

Für diese Projektarbeit liegt der Fokus vorallem auf der
(mathematischen) Beschreibung von Oberflächen und der Nutzung dieser,
sowie der Darstellung von Szenen mittels Sphere Tracing. Bei dem
erstellten Prototypen handelt es sich um eine Art Machbarkeitsstudie,
welche zeigt, dass Bilder mittels Sphere Tracing in Echtzeit erzeugt und
dargestellt werden können.

Diese Projektarbeit dient als Vorarbeit und Grundlage für das MSE-Modul
\textit{MTE7102}.

\subsection{Vorgängige Arbeiten}
\label{subsec:preliminaries}

\todo[inline]{Describe preliminaries}

\subsection{Neue Lerninhalte}
\label{subsec:new_learning_contents}

\todo[inline]{Describe new learning contents}
