% In den Anhang fügen Sie ein:
%  * Details des Projektpans, falls vorhanden
%  * Resultate und Zwischenresultate in Funktion der Projektiterationen
%  * Pflichtenheft / Anforderungsspezifikation (Stand Ende dritter Woche)
%  * Angaben zum Projektrepository
%  * Sitzungsprotokolle, falls vorhanden
%  * Weiterführende Erläuterungen zu den verwendeten Technologien, falls nötig
%  * Benutzerhandbuch, falls vorhanden und sinnvoll, es hier aufzulisten
%  * Installations- und Betriebsdokument, falls vorhanden und sinnvoll, es hier aufzulisten
% Unterlassen Sie das Anfügen von Listings.

\appendix 
\begin{titlepage}
    \clearpage
    \vspace*{\fill}
    \begin{center}
        \begin{minipage}{.6\textwidth}
            \fontsize{26pt}{28pt}\selectfont
            \chapter*{Anhang}\label{chap:attachment}
            \addcontentsline{toc}{chapter}{Anhang}
        \end{minipage}
    \end{center}
    \vfill % equivalent to \vspace{\fill}
    \clearpage
\end{titlepage}
\newpage
% \includepdfset{pagecommand={\thispagestyle{headings}}}
% \includepdf[pages=-, addtotoc={1,chapter,0,Anforderungsdokument,chap:anf},scale=0.95]{anhang/anforderungen.pdf}
% \newpage
% \includepdf[pages=-, addtotoc={1,chapter,0,Tutorial Wissensmodellierung,chap:tutorial},scale=0.95]{anhang/Tutorial.pdf}
% \newpage
% \input{anhang/schnipsel}
% \newpage
% \input{anhang/modellierung}
% \newpage
% \input{anhang/installationshandbuch}
% \newpage
% \includepdf[pages=-, addtotoc={1,chapter,0,Arbeitsjournal ,chap:arbeitsjournal},scale=0.95]{anhang/Journal.pdf}
% \newpage
