% -*- coding: UTF-8 -*-
% vim: autoindent expandtab tabstop=4 sw=4 sts=4 filetype=tex
% chktex-file 27 - disable warning about missing include files

\section{Rendering von impliziten Oberflächen}
\label{sec:rendering_implicit_surfaces}
\todo[inline]{Add an introduction}

\subsection{Beleuchtungsmodell}
\label{sec:rendering_implicit_surfaces_lighting}
\todo[inline]{Explain lighting model}

Um implizite Oberflächen darstellen zu können, ist es notwendig ein
Beleuchtungsmodell zu wählen. Ansonsten wäre das dargestellte Bild nur schwarz.
Der Einfachheit halber wird im Rahmen dieser Projektarbeit das in
Kapitel~\ref{subsec:local_illumination_models} vorgestellte
Phong-Beleuchtungsmodell verwendet.

Daher wird die resultierende Farbe eines Punktes im Raum $I(\bm{x})$ aus
ambienten, diffusen und reflektierenden Anteilen berechnet:

\begin{gather}
    I(\bm{x}) = I_{\text{ambient}} + I_{\text{diffuse}} + I_{\text{specular}}
\end{gather}

Wie bereits zuvor in Kapitel~\ref{subsec:local_illumination_models} erwähnt,
wird der emissive Term bewusst weggelassen, da keine emissiven Materialen
dargestellt werden sollen. Als Lichtquelle wird eine einzelne direktionale
Lichtquelle gewählt. Analog zu den vorherigen Abschnitten ist $\bm{x}$
in den folgenden Abschnitten ein Punkt $(x, y, z)$ auf einer impliziten
Oberfläche $A$.

Der \textit{ambiente Anteil} $I_{\text{ambient}}$ ergibt sich dann wie
folgt:

\begin{gather}
    \text{I}_{\text{ambient}} = k_{\text{ambient}}(\bm{x}) \cdot L_{\text{ambient}}
\end{gather}

wobei $k_{\text{ambient}}(\bm{x})$ den ambienten Faktor des
Punktes $\bm{x}$ und $L_{\text{ambient}}$ die Farbe des eingehenden ambienten
Lichtes ist.

Der \textit{diffuse Anteil} $I_{\text{diffuse}}$ ergibt sich wie folgt:

\begin{gather}
    I_{\text{diffuse}} = k_{\text{diffuse}}(\bm{x}) \cdot L_{\text{diffuse}} \cdot \max(\bm{n} \cdot \bm{l}, 0)
\end{gather}

wobei $k_{\text{diffuse}}(\bm{x})$ den diffusen Faktor am Punkt $\bm{x}$
und $L_{\text{diffuse}}$ die Farbe des eingehenden diffusen Lichtes ist.
Die Richtung der Lichtquelle, ausgehend von Punkt $\bm{x}$, ergibt sich
durch das Punktprodukt zwischen der Einheitsnormalen $\bm{n}$ des
Punktes und dem Einheitsvektor $\bm{l}$.

Der \textit{reflektierende Anteil} $I_{\text{specular}}$ ergibt sich
wie folgt:

\begin{gather}
    I_{\text{specular}} = n_{\text{facing}} \cdot k_{s}(\bm{x}) \cdot L_{\text{specular}} \cdot \max{(\bm{n} \cdot \bm{h}, 0)}^{k_{e}}
\end{gather}

wobei $k_{\text{specular}}(\bm{x})$ den reflektierenden Faktor des
Punktes $\bm{x}$ und $L_{\text{specular}}$ die Farbe des eingehenden
reflektierenden Lichtes ist. Bei $\bm{h}$ handelt es sich um einen
Einheitsvektor, welcher in der Hälfte zwischen der Blickrichtung des
Betrachters bzw.\ der Kamera ($\vv{V}$) und $\bm{l}$ der Richtung der
Lichtquelle ausgehend von dem Punkt $\bm{x}$ ist. Der Exponent $k_{e}$
gibt an, wie rau bzw.\ wie spiegelnd die Oberfläche am Punkt $\bm{x}$
ist. Der Faktor $n_{\text{facing}}$ definiert, ob die Oberfläche
überhaupt einen reflektierenden Anteil hat:

\begin{equation}
    n_{\text{facing}} = \begin{cases}
        0 & \quad \text{if } \bm{n} \cdot \bm{l} \leq 0\\
        1 & \quad \text{if } \bm{n} \cdot \bm{l} > 0 \\
    \end{cases}
\end{equation}

Für die Berechnung der Lichtintensität bzw.\ der Farbe einer Oberfläche
wird die Normale der Oberfläche benötigt. Gemäss~\cite{hart_ray_1989}
kann diese mittels des Gradienten des Distanzfeldes
eines Punktes einer impliziten Oberfläche berechnet werden:

\begin{gather}
    \bm{n}_{x} = f(x + \varepsilon, y, z) - f(x - \varepsilon, y, z) \\
    \bm{n}_{y} = f(x, y + \varepsilon,  z) - f(x, y - \varepsilon,  z) \\
    \bm{n}_{z} = f(x, y, z + \varepsilon) - f(x, y, z - \varepsilon) \\
\end{gather}

wobei $\bm{n} = \begin{bmatrix} x_{n} \\ y_{n} \\ z_{n} \end{bmatrix}$
die Normale der Oberfläche in Form eines Vektors, und $f$ eine
Distanzfunktion ist.

Der Gradient einer Funktion $f: \mathbb{R}^{n} \to \mathbb{R}$ wird wie
folgt berechnet:

\begin{gather}
    \text{grad} f = \nabla f = \begin{bmatrix}
        \frac{\partial f}{\partial x_{1}} \\
        \frac{\partial f}{\partial x_{2}} \\
        \vdots \\
        \frac{\partial f}{\partial x_{n}} 
    \end{bmatrix}\\
    \text{grad} f = f_{x}\bm{i} + f_{y}\bm{j} + f_{z}\bm{k}\\
\end{gather}

\cite{hart_ray_1989} gibt dabei $\varepsilon$ als die minimale
Inkrementation eines (Licht-) Strahles an und definiert diesen als
Sichtbarkeitsfunktion $\Gamma_{\alpha, \delta}$:

\begin{gather}
    \Gamma(d) = \alpha d^{\delta}
\end{gather}

in Abhängikeit der euklidischen Distanz
$d$ des Betrachters / der Kamera zur aktuellen Position des (Licht-)
Strahles:

\begin{gather}
    d = |r_{\bm{n}} - r_{0}|
\end{gather}

wobei $\delta$ ein so genannter ``depth-cueing''-Exponent
(``depth-cueing'' oder auch ``foldback'', ``a process for returning a
signal to a performer
instantly''~\cite{liam_collins_sons_&_co._ltd._collins_2015}) und
$\alpha$ ein empirischer Anteil, welcher die Tiefenauflösung des
Objektes definiert,
ist.
Details dazu finden sich unter~\cite{hart_ray_1989}[S. 293, Abschnitt
4.2 --- ``Clarity''].

Es folgt also:

\begin{gather}
    \varepsilon = \Gamma_{\alpha, \delta}(|r_{\bm{n}} - r_0|)
\end{gather}

Die Korrektheit der Berechnung der Normalen $\bm{n}$ hängt von der
Grösse von $\varepsilon$ ab. Daher wird für gewöhnlich ein kleiner Wert
für $\varepsilon$ gewählt.

Die Normale der Oberfläche sollte schliesslich noch normalisiert werden.

\cite{hart_ray_1989} schreibt weiter, dass die oben genannte Gradiente,
bestehend aus 6 Punkten, durch Hinzunahme von Punkten, welche eine
gemeinsame Kante haben, erweitert werden kann. Dies erzeugt eine
Gradiente bestehend aus 18 Punkten. Werden noch die Punkte
hinzugenommen, welche gemeinsame Eckpunkte haben, so ergibt sich eine
Gradiente bestehend aus 26 Punkten.
Dies macht jedoch nur dann Sinn, wenn die Gradiente mit 6 Punkten eine
unzureichende Genauigkeit liefert.

\subsection{Rendering}
\label{sec:rendering_implicit_surfaces_Rendering}

Um implizite Oberflächen zu rendern werden die in
Abschnitt~\ref{subsec:implicit_surfaces_primitives} angegebenen
Primitiven verwendet.

Zum eigentlichen Rendern wird der Algorithmus~\ref{alg:sphere_tracing}
mit dem unter Abschnitt~\ref{sec:rendering_implicit_surfaces_lighting}
angegebenen Beleuchtungsmodell angwendet.

\subsection{Schatten}
\label{sec:rendering_implicit_surfaces_shadows}
\todo[inline]{Explain shadowing}
