% -*- coding: UTF-8 -*-
% vim: autoindent expandtab tabstop=4 sw=4 sts=4 filetype=tex
% chktex-file 27 - disable warning about missing include files

\section{Rendering von impliziten Oberflächen}
\label{sec:rendering_implicit_surfaces}
\todo[inline]{Add an introduction}

\subsection{Beleuchtungsmodell}
\label{sec:rendering_implicit_surfaces_lighting}
\todo[inline]{Explain lighting model}

Um implizite Oberflächen darstellen zu können, ist es notwendig ein
Beleuchtungsmodell zu wählen. Ansonsten wäre das dargestellte Bild nur schwarz.
Der Einfachheit halber wird im Rahmen dieser Projektarbeit das in
Kapitel~\ref{subsec:local_illumination_models} vorgestellte
Phong-Beleuchtungsmodell verwendet.

Daher wird die resultierende Farbe eines Punktes im Raum $I(\bm{x})$ aus
ambienten, diffusen und reflektierenden Anteilen berechnet:

\begin{gather}
    I(\bm{x}) = \text{Amount}_{\text{ambient}} + \text{Amount}_{\text{diffuse}} + \text{Amount}_{\text{specular}}
\end{gather}

Wie bereits zuvor in Kapitel~\ref{subsec:local_illumination_models} erwähnt,
wird der emissive Term bewusst weggelassen, da keine emissiven Materialen
dargestellt werden sollen. Als Lichtquelle wird eine einzelne direktionale
Lichtquelle gewählt.

Der \textit{ambiente Anteil} $\text{Amount}_{\text{ambient}}$ ergibt sich dann wie
folgt:

\begin{gather}
    \text{Amount}_{\text{ambient}} = k_{\text{ambient}}(\bm{x}) \cdot
                              I_{\text{ambient}}
\end{gather}

wobei $k_{\text{ambient}}(\bm{x})$ den ambienten Faktor des
Punktes $\bm{x}$ und $I_{\text{ambient}}$ die Farbe des eingehenden ambienten
Lichtes ist.

Der \textit{diffuse Anteil} $\text{Amount}_{\text{diffuse}}$ ergibt sich wie folgt:

\begin{gather}
    \text{Amount}_{\text{diffuse}} = k_{\text{diffuse}}(\bm{x}) \cdot
                              I_{\text{diffuse}} \cdot \max(\bm{n} \cdot
                              \bm{l}, 0)
\end{gather}

wobei $\bm{x}$ ein Punkt $(x, y, z)$ auf einer impliziten Oberfläche $A$,
$k_{\text{diffuse}}(\bm{x})$ den diffusen Faktor am Punkt $\bm{x}$ und
$I_{\text{diffuse}}$ die Farbe des eingehenden diffusen Lichtes ist. Das
Punktprodukt zwischen der Einheitsnormalen $\bm{n}$ des Punktes $\bm{x}$ und
dem Einheitsvektor $l$ stellt die Richtung der Lichtquelle ausgehend von dem
Punkt $\bm{x}$ dar.

Der \textit{reflektierende Anteil} $\text{Amount}_{\text{specular}}$ ergibt sich
wie folgt:

\begin{gather}
    \text{Amount}_{\text{specular}} = c_{\text{facing}} \cdot k_{s}(\bm{x}) \cdot I_{\text{specular}} \cdot \max{(\bm{n} \cdot \bm{h}, 0)}^{k_{e}}
\end{gather}

wobei $k_{\text{specular}}(\bm{x})$ den reflektierenden Faktor des Punktes
$\bm{x}$ und $I_{\text{specular}}$ die Farbe des eingehenden reflektierenden
Lichtes ist. Bei $\bm{h}$ handelt es sich um einen Einheitsvektor, welcher in
der Hälfte zwischen der Blickrichtung des Betrachters bzw.\ der Kamera
($\vv{V}$) und $\bm{l}$ der Richtung der Lichtquelle ausgehend von dem Punkt
$\bm{x}$ ist.

\subsection{Rendering}
\label{sec:rendering_implicit_surfaces_Rendering}
\todo[inline]{Explain rendering}

\subsection{Schatten}
\label{sec:rendering_implicit_surfaces_shadows}
\todo[inline]{Explain shadowing}
