% -*- coding: UTF-8 -*-
% vim: autoindent expandtab tabstop=4 sw=4 sts=4 filetype=tex
% vim: spelllang=de spell
% chktex-file 27 - disable warning about missing include files

\subsection{Primitive}
\label{subsec:implicit_surfaces_primitives}

\citeauthor{hart_sphere_1994} führt in seiner Arbeit einige
(geometrische) Primitive auf, welche nachfolgend erläutert
werden~\parencite[S. 540ff]{hart_sphere_1994}.

\subsubsection{Ebene}
\label{ssubsec:implicit_surfaces_primitives_plane}

Die vorzeichenabhängige Distanz zur einer Ebene $P$ mit einer
Einheitsnormalen $\bm{n}$ und Schnittpunkt $\bm{n} \cdot r$
wird folgend definiert:

\begin{gather}
    d(\bm{x}, P) = \bm{x} \cdot \bm{n} - r
\end{gather}

Dabei ist $r$ die relative Positionierung der Ebene  im Raum. 

\subsubsection{Kugel}
\label{ssubsec:implicit_surfaces_primitives_sphere}

Eine Kugel ist als eine Menge von Punkten (Locus) in fixem Abstand eines
gegebenen Punktes. Die von dem Vorzeichen abhängige Distanz zu einer
Kugel $S$, ausgehend vom Ursprung, ist wie folgt:

\begin{gather}
    d(\bm{x}, S) = \|\bm{x}\| - r
\end{gather}

Dabei stellt $r$ den Radius der Kugel dar.

\subsubsection{Zylinder}
\label{ssubsec:implicit_surfaces_primitives_cylinder}

Die Distanz zu einem um die Z-Achse zentrierten Zylinder mit
Einheitsradius wird durch Projektion auf die XY-Ebene und durch Messung
der Distanz zum Einheitskreis berechnet:

\begin{gather}
    d(\bm{x}, Cyl) = \|(x, y)\| - r
\end{gather}

Dabei stellt $r$ den Radius des Zylinders und $\bm{x}$ den Punkt $(x,y,z)$ im
Raum dar.

\subsubsection{Kegel}
\label{ssubsec:implicit_surfaces_primitives_cone}

Die Distanz zu einem Kegel, welcher am Ursprung zentriert und entlang
der Z-Achse orientiert ist, wird wie folgt berechnet:

\begin{gather}
    d(\bm{x}, Cone) = \|(x, y)\| \cdot \cos(\phi) - |z| \cdot \sin(\phi)
\end{gather}

Dabei stellt $\phi$ den Winkel zur Z-Achse dar.

\subsubsection{Torus}
\label{ssubsec:implicit_surfaces_primitives_torus}

Beim Torus handelt es sich um das Produkt zweier Kreise, sowie den Abstand der Kreise:

\begin{gather}
    d(\bm{x}, T) = \|(\|(x, y)\| - R, z)\| - r
\end{gather}

Dabei stellt $R$ den äusseren Radius und $r$ den innerern Radius des
Torus dar.
Der Torus ist am Ursprung zentriert und dreht sich um die Z-Achse.
