\subsection{Operationen für implizite Oberflächen}
\label{subsec:implicit_surfaces_ops}

Um mit impliziten Oberflächen nicht nur einfache Objekte, wie zum Beispiel eine
Kugel, darstellen zu können, möchte man diese auch transformieren können.

Wie~\citeauthor{hart_sphere_1994} beschreibt, werden implizite
Oberflächen durch die Invertierung des Raumes, in welchem sich eine
Oberfläche befindet, transformiert~\parencite[S. 543]{hart_sphere_1994}.
Der Raum, in dem sich eine implizite Oberfläche befindet, ist die Domäne
der impliziten Funktion der Oberfläche.

Sei $T(\bm{x})$ eine Transformation und $f(\bm{x})$ eine Distanzfunktion,
welche eine implizite Oberfläche definiert. Somit ist die transformierte
implizite Oberfläche~\parencite[S. 534]{hart_sphere_1994}:

\begin{gather}
    f(T^{-1}(\bm{x})) = 0
\end{gather}

Bei Transformationen handelt es sich um von dem Vorzeichen abhängige
Distanzfunktionen (\textit{signed distance functions}).

Es werden folgende Arten von Transformationen
unterschieden~\parencite[S. 14]{hart_ray_1993}:
\begin{itemize}
    \item{Distanz-Operationen}\\
        Zum Beispiel Vereinigung, Subtraktion oder Intersektion.
    \item{Domänen-Operationen}\\
        Zum Beispiel Wiederholung, Rotation, Translation und Skalierung.
    \item{Distanz-Deformationen}\\
        Zum Beispiel Versatz (displacement) und Vermengung/Vermischung.
        (\textit{blend})
    \item{Domänen-Deformationen}\\
        Zum Beispiel ``Verdrehung'' (\textit{twist}) und Biegung
        (\textit{bend}).
\end{itemize}

\subsubsection{Isometrien}
\label{ssubsec:implicit_surfaces_ops_isometries}

Nicht alle Transformationen erhalten dabei die Distanz, welche die
Distanzfunktion der transformieren Oberfläche zurückgeben würde. In
solch einem Falle ist die zurückgegebene Distanz nicht die Distanz eines
beliebigen Punktes im Raum zu dem ihm nächsten Punkt einer impliziten
Oberfläche.

Transformationen, welche die Distanz hingegen erhalten,
bezeichnet~\citeauthor{hart_sphere_1994}
als~\textit{Isometrien}~\parencite[S. 534]{hart_sphere_1994}. Dazu
zählen Rotationen, Translationen aber auch Reflexionen.

Ist $\bm{I}$ eine Isometrie, dann benötigt die zurückgegebene Distanz der
Distanzfunktion $f(\bm{x})$ \textit{keine Anpassung}.

\begin{gather}
    d(\bm{x}, \bm{I} \circ f^{-1}(0)) = d(\bm{I}^{-1}(\bm{x}), f^{-1}(0))
\end{gather}

Wobei $\bm{I}$ eine Isometrie und $f^{-1}(0)$ eine implizite Oberfläche ist.

\subsubsection{Uniforme Skalierung}
\label{ssubsec:implicit_surfaces_ops_scaling}

Eine Skalierung erhält die Distanz, welche die Distanzfunktion der
skalierten Oberfläche zurückgeben würde, \textit{nicht}.  Somit muss die
zurückgegebene Distanz entsprechend angepasst werden.

\citeauthor{hart_sphere_1994} gibt die uniforme Skalierung als
Transformation $\bm{S(x)}$  der Form~\parencite[S. 534]{hart_sphere_1994}:

\begin{gather}
    \bm{S(x)} = s \cdot \bm{x}
\end{gather}

an, wobei $s$ der Skalierungsfaktor ist. Die Invertierung der Skalierung
ist gegeben als~\parencite[S. 534]{hart_sphere_1994}:

\begin{gather}
    \bm{S^{-1}(x)} = {1 \over s} \cdot \bm{x}
\end{gather}

Somit ist die Distanz zu der skalierten impliziten
Oberfläche~\parencite[S. 534]{hart_sphere_1994}:

\begin{gather}
    d(\bm{x}, \bm{S}(f^{-1}(0))) = s \cdot d(\bm{S}^{-1}(\bm{x}), f^{-1}(0))
\end{gather}

Dabei wird die von der Distanzfunktion der skalierten impliziten Oberfläche
zurückgegebene Distanz mit dem Skalierungsfaktor $s$ multipliziert, was die
eigentliche Information der Distanz erhält und die Skalierung somit isometrisch
macht.

\subsubsection{``Verdrehung'' (Twist)}
\label{ssubsec:implicit_surfaces_ops_twist}

Gemäss~\citeauthor{hart_sphere_1994} werden bei der ``Verdrehung'' (Twist)
einer impliziten Oberfläche zwei Achsen (z.B. $x$ und $y$) anhand einer
linearen Funktion $a(\cdot)$ in Abhängigkeit der dritten Achse (z.B. $z$)
rotiert~\parencite[S. 543]{hart_sphere_1994}:

\begin{gather}
    twist(\bm{x}) = \begin{pmatrix} 
        x \cdot \cos{a(z)} - y \cdot \sin{a(z)},\\
        x \cdot \sin{a(z)} + y \cdot \cos{a(z)},\\
        z
    \end{pmatrix}
\end{gather}

\subsubsection{Vereinigung}
\label{ssubsec:implicit_surfaces_ops_union}

Die Vereinigung zweier impliziter Oberflächen $A$ und $B$ wird
von~\citeauthor{hart_sphere_1994} als minimale Distanz der jeweiligen,
vom Vorzeichen abhängigen Distanzfunktion $f_{A}$ respektive $f_{B}$
definiert~\parencite[S. 531 bis 532]{hart_sphere_1994}:

\begin{gather}
    d(\bm{x}, A \cup B) = \min(f_{A}(\bm{x}), f_{B}(\bm{x}))
\end{gather}

wobei $\bm{x}$ den abzutastenden Punkt im Raum darstellt.

Wie~\citeauthor{hart_sphere_1994} schreibt, ist die Distanz zu einer Liste von
impliziten Oberflächen die kleinste Distanz der jeweiligen Distanzfunktion.
Somit erlaubt es die Vereinigung --- neben der eigentlichen Vereinigung von
Objekten --- mehrere implizite Oberflächen zu kombinieren, ohne dass diese
miteinander in Kontakt stehen müssen. So kann beispielsweise eine komplexe
Szene modelliert werden.

\subsubsection{Subtraktion}
\label{ssubsec:implicit_surfaces_ops_subtraction}

Um die Operation der Subtraktion zu definieren, wird die Distanz zum
Komplement eines Objektes $\bm{A}$ verwendet. Dabei wird die Eigenschaft
der Abhängigkeit von dem Vorzeichen von entsprechenden Distanzfunktionen
genutzt~\parencite[S. 532]{hart_sphere_1994}:

\begin{gather}
    d(\bm{x}, \mathbb{R}^{3} \setminus A) = -f_{A}(\bm{x})
\end{gather}

Somit kann die Subtraktion zweier impliziter Oberflächen $A$ und $B$
gemäss~\citeauthor{hart_sphere_1994} als Intersektion eines Objektes $A$ mit der
Subtraktion des Raumes bzw.\ der Domäne mit einem Objekt $B$ angesehen werden,
daher folgt~\parencite[S. 532]{hart_sphere_1994}:

\begin{align}
    d(\bm{x}, A - B) &= A \cap (\mathbb{R}^{3} \setminus B) \\
                     &\geq \max(f_{A}(\bm{x}), -f_{B}(\bm{x}))
\end{align}

wobei $\bm{x}$ den abzutastenden Punkt im Raum darstellt.

\subsubsection{Intersektion}
\label{ssubsec:implicit_surfaces_ops_intersection}

Die Intersektion zweier impliziter Oberflächen $A$ und $B$ wird
von~\cite{hart_sphere_1994} als minimale Distanz der jeweiligen
vorzeichenabhängigen  Distanzfunktion $f_{A}$ respektive $f_{B}$
definiert~\parencite[S. 532]{hart_sphere_1994}:

\begin{gather}
    d(\bm{x}, A \cap B) \geq \max(f_{A}(\bm{x}), f_{B}(\bm{x}))
\end{gather}

wobei $\bm{x}$ den abzutastenden Punkt im Raum darstellt.
