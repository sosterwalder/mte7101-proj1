% -*- coding: UTF-8 -*-
% vim: autoindent expandtab tabstop=4 sw=4 sts=4 filetype=tex
% chktex-file 27 - disable warning about missing include files

\chapter{Schlusswort}
\label{chap:discussion_and_conclusion}

In dieser Projektarbeit wurde mit Sphere Tracing eine Methode
vorgestellt, bei welcher es sich um eine optimierte Variante von dem als
Ray Tracing bekannten Verfahren handelt. Diese erlaubt die Darstellung
von Szenen analog dem Ray Tracing Verfahren, aber in Echtzeit.

Zuerst wurden \textit{lokale}, dann \textit{globale Beleuchtungsmodelle} inklusive der
\textit{Renderinggleichung} vorgestellt. Weiter wurde das \textit{Ray Tracing}
Verfahren ausgehend von der ursprünglichen Idee des Ray Tracings, ein
als \textit{Ray Casting} bekanntes Verfahren, vorgestellt. Dabei wurde auch die
zugrundeliegende Physik vereinfacht aufgezeigt.

Für die Anwendung der Beleuchtungsmodelle in einem praktischen Sinne, so
dass eine Szene auf dem Bildschirm sichtbar wird, wurden die klassischen
Modelle zur Schattierung \textit{Flat-Shading}, \textit{Gouraud-Shading}
sowie \textit{Phong-Shading} vorgestellt.

Um die eigentliche Szene darstellen zu können,
wurden~\textit{Oberflächen} im Allgemeinen eingeführt. Dabei wurde
gezeigt, dass zur Modellierung von Oberflächen hauptsächlich zwei
Techniken verwendet werden: Die parametrische und die
implizite Modellierung bzw.\ Darstellung. Es wurden dann
\textit{implizite Oberflächen} im Speziellen behandelt. Mittels
\textit{Distanzfunktionen} wurde eine Grundlage zur Berechnung von
Distanzen gezeigt, welche der späteren Modellierung und Darstellung dient.

Ausgehend von den Distanzfunktionen wurden schliesslich
\textit{Distanzfelder} vorgestellt, eine Art Datenstruktur basierend auf
den Distanzfunktionen.

Mit \textit{Ray Marching} und \textit{Sphere Tracing} wurden zwei Methoden
zur Darstellung von impliziten Oberflächen aufgezeigt. Weiter wurden
\textit{Operationen für implizite Oberflächen}, wie beispielsweise die
Vereinigung oder Subtraktion, vorgestellt, sowie eine Menge von
Funktionen zur Modellierung von \textit{geometrischen Primitiven}.

Es wurde schliesslich gezeigt, wie die vorgestellten Grundalgen für das
Rendering von impliziten Oberflächen genutzt werden können.

Im letzten Abschnitt wurde der umgesetzte \textit{Prototyp} vorgestellt,
wobei die Parameter \textit{Distanz}, \textit{Präzision}, die
\textit{Anzahl Schritte} und der \textit{Skalierungsfaktor für Scahtten}
anhand einer Beispielszene verglichen wurden.

\section{Erweiterungsmöglichkeiten}
\label{sec:further_work}

Enhanced Sphere Tracing, Cone Tracing, FXAA.\
