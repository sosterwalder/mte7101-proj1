% -*- coding: UTF-8 -*-
% vim: autoindent expandtab tabstop=4 sw=4 sts=4 filetype=tex
% chktex-file 27 - disable warning about missing include files

\chapter{Schlusswort}
\label{chap:discussion_and_conclusion}

In dieser Projektarbeit wurde mit Sphere Tracing eine Methode
vorgestellt, bei welcher es sich um eine optimierte Variante von dem als
Ray Tracing bekannten Verfahren handelt. Diese erlaubt die Darstellung
von Szenen analog dem Ray Tracing Verfahren, aber in Echtzeit.

Zuerst wurden \textit{lokale}, dann \textit{globale Beleuchtungsmodelle} inklusive der
\textit{Renderinggleichung} vorgestellt. Weiter wurde das \textit{Ray Tracing}
Verfahren ausgehend von der ursprünglichen Idee des Ray Tracings, ein
als \textit{Ray Casting} bekanntes Verfahren, vorgestellt. Dabei wurde auch die
zugrundeliegende Physik vereinfacht aufgezeigt.

Für die Anwendung der Beleuchtungsmodelle in einem praktischen Sinne, so
dass eine Szene auf dem Bildschirm sichtbar wird, wurden die klassischen
Modelle zur Schattierung \textit{Flat-Shading}, \textit{Gouraud-Shading}
sowie \textit{Phong-Shading} vorgestellt.

Um die eigentliche Szene darstellen zu können,
wurden~\textit{Oberflächen} im Allgemeinen eingeführt. Dabei wurde
gezeigt, dass zur Modellierung von Oberflächen hauptsächlich zwei
Techniken verwendet werden: Die parametrische und die
implizite Modellierung bzw.\ Darstellung. Es wurden dann
\textit{implizite Oberflächen} im Speziellen behandelt. Mittels
\textit{Distanzfunktionen} wurde eine Grundlage zur Berechnung von
Distanzen gezeigt, welche der späteren Modellierung und Darstellung dient.

Ausgehend von den Distanzfunktionen wurden schliesslich
\textit{Distanzfelder} vorgestellt, eine Art Datenstruktur basierend auf
den Distanzfunktionen.

Mit \textit{Ray Marching} und \textit{Sphere Tracing} wurden zwei Methoden
zur Darstellung von impliziten Oberflächen aufgezeigt. Weiter wurden
\textit{Operationen für implizite Oberflächen}, wie beispielsweise die
Vereinigung oder Subtraktion, vorgestellt, sowie eine Menge von
Funktionen zur Modellierung von \textit{geometrischen Primitiven}.

Es wurde schliesslich gezeigt, wie die vorgestellten Grundalgen für das
Rendering von impliziten Oberflächen genutzt werden können.

Im letzten Abschnitt wurde der umgesetzte \textit{Prototyp} vorgestellt,
wobei die Parameter \textit{Distanz}, \textit{Präzision}, die
\textit{Anzahl Schritte} und der \textit{Skalierungsfaktor für Scahtten}
anhand einer Beispielszene verglichen wurden.

\section{Erweiterungsmöglichkeiten}
\label{sec:further_work}

Da diese Projektarbeit einen begrenzten Zeitrahmen von einem Semester
hat, konnten nicht alle gewünschten Thematiken bearbeitet werden.

Zudem handelt es sich bei Sphere Tracing um ein relativ junges
Verfahren, welches noch über viel Potential für Forschung und
Entwicklung verfügt. Da es sich zudem um ein grosses Themengebiet
handelt, kann hier auch entsprechend viel Zeit investiert werden.

Einige der während der Recherche aufgekommenen, aber nicht bearbeiteten
Thematik werden nachfolgend aufgezeigt.

\subsection{Umgebungs-Beleuchtung bzw.\ -Verdeckung (Ambient Occlusion)}
\label{subsec:further_work:ambient_occlusion}

``Umgebungsverdeckung (englisch Ambient Occlusion, AO) ist eine
Shading-Methode, die in der 3D-Computergrafik verwendet wird, um mit
relativ kurzer Renderzeit eine realistische Verschattung von Szenen zu
erreichen. Das Ergebnis ist zwar nicht physikalisch korrekt, reicht
jedoch in seinem Realismus oft aus, um auf rechenintensive globale
Beleuchtung verzichten zu
können.''~\parencite{wikipedia_the_free_encyclopedia_umgebungsverdeckung_2015}

\citeauthor{evans_fast_2006} liefert in seiner
Arbeit~\citetitle{evans_fast_2006} interessante Ansätze, wie die
Umgebungsverdeckung auf Distanzfelder angewendet werden kann. Dies kann
so wohl mit relativ wenig Aufwand auch für Sphere Tracing angewendet
werden~\parencite{evans_fast_2006}.

\subsection{Kantenglättung}
\label{subsec:further_work:antialiasing}

Bei der Darstellung von Szenen mittels Sphere Tracing kommt es an Kanten
zu harten Übergängen und Artefakten --- es tritt das so genannte
``Aliasing'' auf.

\citeauthor{hart_sphere_1994} schlägt in~\citetitle{hart_sphere_1994}
bereits Verfahren zur Kantenglättung vor. Diese verkomplizieren jedoch
den Rendering-Prozess und haben schwächen im Umgang mit angenäherten
Obergrenzen von impliziten Oberflächen

\citeauthor{lottes_fxaa_2009} schlägt in seiner
Arbeit~\citetitle{lottes_fxaa_2009} eine einfache Lösung dieser
Problematiken vor. Es handelt sich dabei um ein Einpass
Filter, welcher typischerweise in der Nachbearbeitung zum Einsatz kommt.
Der Filter wird auf das gerenderte Bild (bzw.\ auf einen Bildpuffer)
angewendet. Es handelt sich dabei um ein Verfahren, welches auf der
Detektion von Kanten basiert~\parencite{lottes_fxaa_2009}.

\subsection{Realistische Materialien}
\label{subsec:further_work:brdf}

In~\autoref{subsec:ray_tracing} werden Verfahren zur Berechnung der
physikalischen Eigenschaften von Oberflächen und damit Materialien
aufgezeigt. Dabei handelt es sich aber nur um Näherungen unter Annahme
der perfekten Bedingungen.

Eine allgemeinere und daher realistischerer Form ist die so genannte
BRDF:\ ``Eine bidirektionale Reflektanzverteilungsfunktion (engl.
Bidirectional Reflectance Distribution Function, BRDF) stellt eine
Funktion für das Reflexionsverhalten von Oberflächen eines Materials
unter beliebigen Einfallswinkeln dar. Sie liefert für jeden auf dem
Material auftreffenden Lichtstrahl mit gegebenem Eintrittswinkel den
Quotienten aus Strahlungsdichte und Bestrahlungsstärke für jeden
austretenden Lichtstrahl. BRDFs werden unter anderem in der
realistischen 3D-Computergrafik verwendet, wo sie einen Teil der
fundamentalen Rendergleichung darstellen und dazu dienen, Oberflächen
möglichst realistisch und physikalisch korrekt darzustellen. Eine
Verallgemeinerung der BRDF auf Texturen stellt die BTF (Bidirectional
Texturing Function)
dar.''~\parencite{wikipedia_the_free_encyclopedia_bidirektionale_2014}

\citeauthor{burley_physicall-based_2012} liefert
mit~\citetitle{burley_physicall-based_2012} eine gute Übersicht,
wie solche Verfahren umgesetzt werden
können~\parencite{burley_physicall-based_2012}.
\citeauthor{bagher_accurate_2012} bietet
in~\citetitle{bagher_accurate_2012} eine Fülle an Materialien und
Eigenschaften~\parencite{bagher_accurate_2012}.

\subsection{Optimierungsverfahren}
\label{subsec:further_work:optimisation}

Zur Beschleunigung das Sphere Tracing Verfahrens existieren diverse
Verfahren. So stellen z.B.~\citeauthor{keinert_enhanced_2014}
in~\citetitle{keinert_enhanced_2014} diverse Methoden zur
Optimierung und Beschleunigung des Renderings sowie Methoden zum Finden
von Schnittpunkten vor~\parencite{keinert_enhanced_2014}.~\citeauthor{seven_rendering_2012} stellt
in~\citetitle{seven_rendering_2012} eine Methode vor, wie Sphere Tracing
mittels Aussenden von Kegelförmigen (Primär-) Strahlen optimiert werden
kann~\parencite{seven_rendering_2012}.
