% -*- coding: UTF-8 -*-
% vim: autoindent expandtab tabstop=4 sw=4 sts=4 filetype=tex
% vim: spelllang=de
% chktex-file 27 - disable warning about missing include files

\chapter{Einleitung}
\label{chap:10_introduction}

Seit Bestehen moderner Computer existiert auch die Computergrafik.
Eines ihrer Ziele ist die Projektion des dreidimensionalen Raumes auf
eine zweidimensionale Fläche. In der Regel ist die Ausgabe auf einen
zweidimensionalen Raum beschränkt.

Bei der Ausgabe wird zwischen statischen und dynamischen Bildern
unterschieden. Während statische Bilder konstant sind, können sich
dynamische Bilder ständig ändern. Aufgrund der Reaktionszeiten des
menschlichen Auges müssen für eine ruckfreie Darstellung mindestens 25
Bilder pro Sekunde berechnet werden.
Das Fachgebiet ``Computergrafik'' war schon immer bestrebt eine
möglichst realitätsnahe Darstellung zu erhalten. Eine Darstellung, die
möglichst nahe an der menschlichen Wahrnehmung liegt.

Im Laufe der Zeit entstanden verschiedene Ansätze um eine naturgetreue
Darstellung zu erreichen. Einer der Ansätze sind Beleuchtungsmodelle,
welche die Beleuchtung einer Darstellung bzw.\ einer Szene berechnen.
Dabei wird zwischen lokalen und globalen Beleuchtungsmodellen
unterschieden.

Ein globales Beleuchtungsmodell ist Ray Tracing (zu Deutsch etwa
``Strahlenverfolgung''), welches seinen Ursprung in der
von~\citeauthor{appel_techniques_1968}~\citeyear{appel_techniques_1968}
vorgestellten Arbeit~\citetitle{appel_techniques_1968} hat.
\citeyear{whitted_improved_1980} wurde das Verfahren
von~\citeauthor{whitted_improved_1980}
in~\citetitle{whitted_improved_1980} verbessert.

Das Verfahren besticht durch Einfachheit und bietet eine hohe
Bildqualität mit perfekten Spiegelungen und Transparenzen. Mit
entsprechenden Optimierungen ist das Verfahren schneller geworden, aber
nicht ausreichend.

Mit ``schnell'' ist die Zeitdauer für die Darstellung eines einzelnen
Bildes gemeint. Für eine Darstellung in Echtzeit ist auch das
verbesserte Verfahren noch zu langsam.

Dank Weiterentwicklung der Computer, vor allem der Grafikkarten (GPUs),
ist Ray Tracing wieder in den Fokus der Darstellung von Szenen in
Echtzeit gerückt.

Diese Projektarbeit stellt ein spezielles Ray Tracing Verfahren zur
Darstellung von Bildern in Echtzeit vor: \textit{Volume Ray Casting}
bzw.~\textit{Sphere Tracing} genannt.
