% -*- coding: UTF-8 -*-
% vim: autoindent expandtab tabstop=4 sw=4 sts=4 filetype=tex
% chktex-file 27 - disable warning about missing include files

% Main document
% ===========================================================================
% This is part of the document "Project documentation template".
% Authors: brd3, kaa1
%
%---------------------------------------------------------------------------

\documentclass[
    a4paper,                % paper format
    10pt,                   % fontsize
    %twoside,               % double-sided
    openright,              % begin new chapter on right side
    notitlepage,            % use no standard title page
    parskip=half,           % set paragraph skip to half of a line
]{scrreprt}                 % KOMA-script report
%---------------------------------------------------------------------------
\raggedbottom{}
%\KOMAoptions{cleardoublepage=plain}         % Add header and footer on blank pages


% Load Standard Packages:
%---------------------------------------------------------------------------
\usepackage{scrpage2}
\usepackage[ngerman]{babel}                                             % german hyphenation
\usepackage[utf8]{inputenc}                                             % UTF-8 input encoding
\usepackage[T1]{fontenc}                                                % hyphenation of words with ä,ö and ü
\usepackage{textcomp}                                                   % additional symbols
\usepackage{etoolbox}                                                   % color manipulation of header and footer
\usepackage{graphicx}                                                   % integration of images
\usepackage{float}                                                      % floating objects
\usepackage[font={footnotesize,it}]{caption}                            % for captions of figures and tables
\usepackage{booktabs}                                                  % package for nicer tables
% \usepackage{tabu}                                                       % package for tables
\usepackage{tocvsec2}                                                   % provides means of controlling the sectional numbering
\usepackage[square,sort,comma,authoryear]{natbib}                       % provides various citation styles
\usepackage{wrapfig}                                                    % provides floating of text around images
\usepackage{nameref}                                                    % provides printing names of references
\usepackage{colortbl}                                                   % Colored tables
\usepackage{scrhack}                                                    % Remove float errors and warnings
\usepackage{pgfgantt}                                                   % Provides GANTT charts
\usepackage{array}
%---------------------------------------------------------------------------

% Load Math Packages
%---------------------------------------------------------------------------
\usepackage[fleqn]{amsmath}                                             % various features to facilitate writing math formulas
\usepackage{amsthm}                                                     % enhanced version of latex's newtheorem
\usepackage{amsfonts}                                                   % set of miscellaneous TeX fonts that augment the standard CM
\usepackage{amssymb}                                                    % mathematical special characters
\usepackage{exscale}                                                    % mathematical size corresponds to textsize
%---------------------------------------------------------------------------

% Package to facilitate placement of boxes at absolute positions
%---------------------------------------------------------------------------
\usepackage[absolute]{textpos}
\setlength{\TPHorizModule}{1mm}
\setlength{\TPVertModule}{1mm}
%---------------------------------------------------------------------------

% PDF as attachment
%---------------------------------------------------------------------------
\usepackage{pdfpages}
%---------------------------------------------------------------------------

% Definition of Colors
%---------------------------------------------------------------------------
\RequirePackage{color}                          % Color (not xcolor!)
\definecolor{linkblue}{rgb}{0,0,0.8}            % Standard
\definecolor{darkblue}{rgb}{0,0.08,0.45}        % Dark blue
\definecolor{bfhgrey}{rgb}{0.41,0.49,0.57}      % BFH grey
%\definecolor{linkcolor}{rgb}{0,0,0.8}              % Blue for the web- and cd-version!
\definecolor{linkcolor}{rgb}{0,0,0}                 % Black for the print-version!
\colorlet{Black}{black}
\definecolor{keywords}{rgb}{255,0,0}
\definecolor{red}{rgb}{0.6,0,0}
\definecolor{green}{rgb}{0,0.5,0}
\definecolor{blue}{rgb}{0,0,0.5}

%---------------------------------------------------------------------------

% Load listings package
% which provides source code formatting
%---------------------------------------------------------------------------
\usepackage{listings}                                                   % provides source code formatting
% Define XML colors
\lstdefinelanguage{XML}
{
  basicstyle=\ttfamily\footnotesize,
  morestring=[b]'',
  moredelim=[s][\bfseries\color{maroon}]{<}{\ },
  moredelim=[s][\bfseries\color{maroon}]{</}{>},
  moredelim=[l][\bfseries\color{maroon}]{/>},
  moredelim=[l][\bfseries\color{maroon}]{>},
  morecomment=[s]{<?}{?>},
  morecomment=[s]{<!--}{-->},
  commentstyle=\color{codecommentcolor},
  stringstyle=\color{darkblue},
  identifierstyle=\color{red}
}
% Change captions of listings
\renewcommand{\lstlistingname}{Auflistung}
\renewcommand{\lstlistlistingname}{Auflistungsverzeichnis}
%---------------------------------------------------------------------------

% Hyperref Package (Create links in a pdf)
%---------------------------------------------------------------------------
\usepackage[
    pdftex,ngerman,bookmarks,plainpages=false,pdfpagelabels,
    backref = {false},                                      % No index backreference
    colorlinks = {true},                  % Color links in a PDF
    hypertexnames = {true},               % no failures "same page(i)"
    bookmarksopen = {true},               % opens the bar on the left side
    bookmarksopenlevel = {0},             % depth of opened bookmarks
    pdftitle = {Volume ray casting --- basics \& principles},      % PDF-property
    pdfauthor = {Sven Osterwalder},                           % PDF-property
    pdfsubject = {Volume ray casting},        % PDF-property
    linkcolor = {linkcolor},              % Color of Links
    citecolor = {linkcolor},              % Color of Cite-Links
    urlcolor = {linkcolor},               % Color of URLs
]{hyperref}
%---------------------------------------------------------------------------

% Geometry package: Set up page dimension
%---------------------------------------------------------------------------
\usepackage[a4paper,
    left=25mm,
    right=25mm,
    top=27mm,
    headheight=20mm,
    headsep=10mm,
    textheight=242mm,
    footskip=15mm
]{geometry}
%---------------------------------------------------------------------------

% Makeindex Package
%---------------------------------------------------------------------------
\usepackage{makeidx}                                % To produce index
\makeindex                                      % Index-Initialisation
%---------------------------------------------------------------------------

% Glossary Package
%---------------------------------------------------------------------------
% the glossaries package uses makeindex
% if you use TeXnicCenter do the following steps:
%  - Goto "Ausgabeprofile definieren" (ctrl + F7)
%  - Select the profile "LaTeX => PDF"
%  - Add in register "Nachbearbeitung" a new "Postprozessoren" point named Glossar
%  - Select makeindex.exe in the field "Anwendung" ( ..\MiKTeX x.x\miktex\bin\makeindex.exe )
%  - Add this [ -s "%tm.ist" -t "%tm.glg" -o "%tm.gls" "%tm.glo" ] in the field "Argumente"
%
% for futher informations go to http://ewus.de/tipp-1029.html
%---------------------------------------------------------------------------
\usepackage[nonumberlist]{glossaries}
%\usepackage[xindy,nonumberlist]{glossaries}
\newglossaryentry{OWL}{name={OWL},description={
    Web Ontology Language;
    Ontologiesprache für das semantische Web.
    Mit dieser Sprache können Ontologien beschrieben werden.
}}

\makeglossaries{}
%---------------------------------------------------------------------------

% Fancyrb package
%---------------------------------------------------------------------------
\usepackage{fancyvrb}
\RecustomVerbatimCommand{\VerbatimInput}{VerbatimInput}
{fontsize=\footnotesize,
    %
    frame=lines,  % top and bottom rule only
    framesep=2em, % separation between frame and text rulecolor=\color{Gray},
    %
    label=\fbox{\color{Black}},
    labelposition=topline,
    %
    % commandchars=\|\(\), % escape character and argument delimiters for
    % commands within the verbatim
    % commentchar=*        % comment character
}
%---------------------------------------------------------------------------

% TODO notes package
%---------------------------------------------------------------------------
\usepackage[textwidth=65mm]{todonotes}
%---------------------------------------------------------------------------

% Intro:
%---------------------------------------------------------------------------
%\begin{document}                                % Start Document
\settocdepth{section}                                                       % Set depth of toc
\pagenumbering{roman}                                                       
%---------------------------------------------------------------------------

\providecommand{\titel}{Volume ray casting --- basics \& principles}
                  % Titel der Arbeit aus Datei titel.tex lesen
% -*- coding: UTF-8 -*-
% vim: autoindent expandtab tabstop=4 sw=4 sts=4 filetype=tex
% chktex-file 27

\providecommand{\versionnumber}{0.1}		%  Hier die aktuelle Versionsnummer eingeben
\providecommand{\versiondate}{{\today}}		%  Hier das Datum der aktuellen Version eingeben
                % Versionsnummer und -datum aus Datei version.tex lesen

% Set up header and footer
%---------------------------------------------------------------------------

\deftripstyle{newlayout}
  [0pt] % no header line
  [0pt] % no footer line
  {}
  {}
  {}
  {\color{bfhgrey} \footnotesize \titel, Version \versionnumber, \versiondate}
  {}
  {\color{bfhgrey} \thepage}

\pagestyle{newlayout}
% use "pagestyle" also on chapter starting pages 
\renewcommand{\chapterpagestyle}{newlayout}
\renewcommand{\chaptermark}[1]{\markboth{\thechapter.  #1}{}}
\renewcommand*{\headfont}{\normalfont}
\renewcommand*{\footfont}{\normalfont}
%---------------------------------------------------------------------------

% We need this as mr. gantt chart (teh package..) thinks text should be gray..
\color{black}

\begin{document}

% Title Page and Abstract
%---------------------------------------------------------------------------
% -*- coding: UTF-8 -*-
% vim: autoindent expandtab tabstop=4 sw=4 sts=4 filetype=tex
% chktex-file 27 - disable warning about missing include files
% chktex-file 36 - disable put space in front of parentheses warning

% Project documentation template
% ===========================================================================
% This is part of the document "Project documentation template".
% Authors: brd3, kaa1
%

\begin{titlepage}


% BFH-Logo absolute placed at (28,12) on A4 and picture (16:9 or 15cm x 8.5cm)
% Actually not a realy satisfactory solution but working.
%---------------------------------------------------------------------------
\setlength{\unitlength}{1mm}
\begin{textblock}{20}[0,0](28,12)
    \includegraphics[scale=1.0]{img/BFH_Logo_B.png}
\end{textblock}

\begin{textblock}{154}(28,48)
    \begin{picture}(150,2)
        \put(0,0){\color{bfhgrey}\rule{150mm}{2mm}}
    \end{picture}
\end{textblock}

\begin{textblock}{154}[0,-0.2](26,40)
    \centering
    \includegraphics[scale=0.6]{img/logo.png}
\end{textblock}

\begin{textblock}{154} (28,135)
    \begin{picture}(150,2)
        \put(0,0){\color{bfhgrey}\rule{150mm}{2mm}}
    \end{picture}
\end{textblock}
\color{black}

% Institution / Titel / Untertitel / Autoren / Experten:
%---------------------------------------------------------------------------
\begin{flushleft}

    \vspace*{120mm}

    \fontsize{26pt}{28pt}\selectfont
    % Titel aus der Datei vorspann/titel.tex lesen
    \titel{}\\
    \vspace{3mm}
    \fontsize{14pt}{16pt}\selectfont
    \textbf{Projektarbeit 1} \\
    \vspace{6mm}

    \textbf{MTE7101} \\
    \vspace{3mm}

    \begin{textblock}{150} (28,215)
        \fontsize{10pt}{17pt}\selectfont
        \begin{tabbing}
        xxxxxxxxxxxxxxx   \= xxxxxxxxxxxxxxxxxxxxxxxxxxxxxxxxxxxxxxxxxxxxxxx \kill
        Studiengang:      \> Informatik                                         \\
        Autor:            \> Sven Osterwalder\protect\footnotemark[1]{}         \\
        Betreuer:         \> Prof.~Claude Fuhrer\protect\footnotemark[2]{} \\
        Datum:            \> \versiondate{}                                     \\
        \end{tabbing}
    \end{textblock}
\end{flushleft}
\footnotetext[1]{sven.osterwalder@students.bfh.ch}
\footnotetext[2]{claude.fuhrer@bfh.ch}

\begin{textblock}{150} (28,280)
\noindent 
\color{bfhgrey}\fontsize{9pt}{10pt}\selectfont
Berner Fachhochschule | Haute école spécialisée bernoise | Bern University of Applied Sciences
\color{black}\selectfont
\end{textblock}


\vfill
\includegraphics[height=\baselineskip]{img/by-sa}\\ \small{\sffamily{Licensed under the Creative Commons Attribution-ShareAlike 3.0 License}}

\end{titlepage}

%
% ===========================================================================
% EOF
%
          % activate for Titelseite mit Bild
\cleardoublepage{}
\phantomsection{}
% Versionenkontrolle :
% -----------------------------------------------

\chapter*{}
\label{chap:versionen}

\begin{textblock}{180} (15,150)
\color{black}
\begin{huge}
Versionen
\end{huge}
\vspace{10mm}

\fontsize{10pt}{18pt}\selectfont
\begin{tabbing}
xxxxxxxxxxx\=xxxxxxxxxxxxxxx\=xxxxxxxxxxxxxxxxxx\=xxxxxxxxxxxxxxxxxxxxxxxxxxxxxxxxxxxxxxxxxxxxxxx     \kill
\textbf{Version}    \> \textbf{Datum}   \> \textbf{Status}      \> \textbf{Bemerkungen}               \\
0.1                 \> 25.09.2015       \> Entwurf              \> Initiale Erstellung des Dokumentes \\
\end{tabbing}

\end{textblock}

\phantomsection{}
\cleardoubleemptypage{}
\listoftodos{}
\phantomsection{}
\cleardoubleemptypage{}
\setcounter{page}{1}
\cleardoublepage{}
\phantomsection{}
\addcontentsline{toc}{chapter}{Management Summary}
\include{inc/static/management_summary}
\cleardoubleemptypage{}
%---------------------------------------------------------------------------

% Make sure Umlauts are getting displayed correctly.
\lstset{literate=%
    {Ö}{\textcolor{black}{\"O}}1
    {Ä}{{\"A}}1
    {Ü}{{\"U}}1
    {ß}{{\ss}}1
    {ü}{{\"u}}1
    {ä}{{\textcolor{black}{\"a}}}1
    {ö}{{\textcolor{black}{\"o}}}1
    {~}{{\textasciitilde}}1
    {?}{{\textcolor{black}{?}}}1
}
% Define a list style for Python programming language
\newcommand\pythonstyle{\lstset{language=Python,
        basicstyle=\ttm,
        keywordstyle=\ttb\color{blue},
        emph={ray_cast},
        emphstyle=\ttb\color{red},
        style,
        frame=tb,
        commentstyle=\color{green},
        showstringspaces=false,
        identifierstyle=\color{black},
        procnamekeys={def,class},
        otherkeywords={self}}}
% Define an environment for Python programming language
\lstnewenvironment{python}[1][]
{
\pythonstyle{}
\lstset{#1}
}
{}

% Table of contents
%---------------------------------------------------------------------------
\tableofcontents
\cleardoublepage{}
%---------------------------------------------------------------------------

% Main part:
%---------------------------------------------------------------------------
\pagenumbering{arabic}
% -*- coding: UTF-8 -*-
% vim: autoindent expandtab tabstop=4 sw=4 sts=4 filetype=tex
% chktex-file 27 - disable warning about missing include files

\chapter{Einleitung}
\label{chap:10_introduction}

Seit dem Bestehen moderner Computer existiert auch die Computergrafik. Ziel der Computergrafik ist es unter Anderem den dreidimensionalen Raum auf eine zweidimensionale Fläche abzubilden, da die Ausgabe meist auf den zweidimensionalen Raum limitiert ist.

Dabei wird zwischen statischen Bildern und dynamischen Bildern unterschieden. Statische Bilder werden bei Bedarf dargestellt und ändern sich in der Regel nicht. Dynamische Bilder können sich hingegen ständig ändern und müssen --- bedingt durch das menschliche Auge --- mit 25 Bildern pro Sekunde ausgegeben werden. Es bestand bereits früh das Bestreben möglichst eine realistische Darstellung zu erhalten. Eine Darstellung also, die möglichst nahe an der menschlichen Wahrnehmung liegt.

Im Laufe der Zeit entstanden verschiedene Ansätze um eine solche Darstellung zu bieten. Ein Teilgebiet davon sind Beleuchtungsmodelle, welche die Beleuchtung einer Darstellung bzw.\ einer Szene berechnen. Dabei wird zwischen lokalen und globalen Beleuchtungsmodellen unterschieden.

Ein globales Beleuchtungsmodell ist Ray Tracing (zu deutsch Strahlenverfolgung), welches 1980 von Turner Whitted vorgestellt wurde wurde. Das Verfahren besticht durch seine Einfachheit und bietet dabei eine hohe Bildqualität mit perfekten Spiegelungen und Transparenzen. Mit entsprechenden Optimierungen ist das Verfahren auch relativ schnell.

Mit schnell ist dabei die Zeit gemeint, die benötigt wird um ein einzelnes Bild darzustellen. Möchte man jedoch eine Darstellung in Echtzeit erreichen, so war das Verfahren lange zu langsam.

Im Rahmen der Weiterentwicklung der Computer und vor allem durch die Weiterentwicklung der Grafikkarten (GPUs) ist Ray Tracing jedoch wieder in den Fokus der Darstellung von Szenen in Echtzeit gerückt.

Diese Projektarbeit stellt ein spezielles, auf Ray Tracing basierendes Verfahren zur Darstellung eines Bildes in Echtzeit vor: Das so Volume Ray Casting oder Sphere Tracing genannte Verfahren.

% -*- coding: UTF-8 -*-
% vim: autoindent expandtab tabstop=4 sw=4 sts=4 filetype=tex
% chktex-file 27 - disable warning about missing include files

\chapter{Administratives}
\label{chap:20_administrative}

\section{Beteiligte Personen}
\label{sec:involved_persons}

\section{Aufbau des Dokumentes}
\label{sec:document_structure}

\section{Ergebnisse}
\label{sec:deliverables}

% -*- coding: UTF-8 -*-
% vim: autoindent expandtab tabstop=4 sw=4 sts=4 filetype=tex
% chktex-file 27 - disable warning about missing include files

\chapter{Aufgabenstellung}
\label{chap:scope}

\todo[inline]{Describe scope}

\section{Motivation}
\label{sec:motivation}

\todo[inline]{Describe motivation}

\subsection{Demoszene}
\label{subsec:demoscene}

\todo[inline]{Loose some words about demoscene! }

\section{Ausgangslage}
\label{sec:initial_situation}

\todo[inline]{Describe initial situation}

\section{Ziele und Abgrenzung}
\label{sec:objectives}

\todo[inline]{Describe objectives}

\subsection{Vorgängige Arbeiten}
\label{subsec:preliminaries}

\todo[inline]{Describe preliminaries}

\subsection{Neue Lerninhalte}
\label{subsec:new_learning_contents}

\todo[inline]{Describe new learning contents}

% -*- coding: UTF-8 -*-
% vim: autoindent expandtab tabstop=4 sw=4 sts=4 filetype=tex
% chktex-file 27 - disable warning about missing include files

\chapter{Vorgehen}
\label{chap:procedure}

\section{Arbeitsorganisation}
\label{sec:organization}

\subsection{Regelmässige Treffen}
\label{subsec:meetings}

Regelmässige Besprechungen mit dem Betreuer der Arbeit halfen die
gesteckten Ziele zu erreichen und Fehlentwicklungen zu vermeiden. Der
Betreuer unterstützte den Autor dabei mit Vorschlägen. Die Treffen
fanden mindestens alle zwei Wochen statt, sie wurden in Form eines
Protokolles festgehalten. Das Protokoll findet sich in
Abschnitt~\ref{chap:10_meeting_minutes} im Anhang dieser Arbeit
(\ref{chap:attachment}).

\section{Projektphasen}
\label{sec:project_schedule}

\subsection{Meilensteine}
\label{subsec:milestones}

Um bei der Arbeit ein möglichst strukturiertes Vorgehen einzuhalten, wurden folgende Projektphasen gewählt:
\begin{itemize}
    \item Start der Projektarbeit
    \item Erarbeitung und Festhalten der Anforderungen
    \item Erarbeitung der theoretischen Grundlagen
    \item Erstellung der abschliessenden Dokumentation
    \item Erstellung eines Prototypen
\end{itemize}

Die Phasen \textit{Erarbeitung der theoretischen Grundlagen}, \textit{Erstellung
der abschliessenden Dokumentation} sowie \textit{Erstellung eines
Prototypen} liefen parallel ab. Erkenntnisse einer Phase flossen jeweils in die
anderen Phasen ein.

\subsection{Zeitplan / Projektphasen}
\label{subsec:timeschedule}

\begin{figure}[H]
    \begin{ganttchart}[
        vgrid,
        x unit=0.5cm,
        bar/.append style={fill=bfhgrey!50},
    ]{1}{22}
        \gantttitle{2015}{15}
        \gantttitle{2016}{7} \ganttnewline{}
        \gantttitlelist{38,...,52,1,2,3,4,5,6,7}{1} \ganttnewline{} % chktex 11: Disable "you should use \ldots to achieve.."
        % \gantttitlelist{1,...,21}{1} \\ % chktex 11: Disable "you should use \ldots to achieve.."
        \ganttbar{Projektstart}{1}{1} \ganttnewline{}
        \ganttlinkedbar{Anforderungen}{2}{3} \ganttnewline{}
        \ganttbar{Erarbeitung Theorie}{2}{10} \ganttnewline{}
        \ganttbar{Erstellung Prototyp}{6}{7} \ganttnewline{}
        \ganttbar{Dokumentation}{1}{10} \ganttnewline{}
        \ganttlinkedbar{Korrekturen}{9}{13} \ganttnewline{}
        \ganttmilestone{Rohfassung Dokumentation}{13} \ganttnewline{}
        \ganttbar{Korrekturen}{14}{19} \ganttnewline{}
        \ganttbar{Vorbereitung Präsentation/Verteidigung}{20}{21} \ganttnewline{}
        \ganttlinkedbar{Präsentation/Verteidigung}{22}{22} \ganttnewline{}
        \ganttmilestone{Abgabe Dokumentation}{22}
    \end{ganttchart}
    \caption{Zeitplan; Der Titel stellt Jahreszahlen, der Untertitel
    Kalenderwochen dar}\label{fig:timeschedule}
\end{figure}

\subsubsection{Projektstart}
\label{subsubsec:kick_off}

In der ersten Phase wurden die Meilensteine der Arbeit identifiziert und
skizziert. Um Details der Aufgabe zu verstehen, wurde das notwendige
Vorwissen über Algorithmen zur globalen Beleuchtung erarbeitet. Weiter wurde
die Grundlage dieser Dokumentation erstellt.

\subsubsection{Anforderungen}
\label{ssubsec:requirements}

In dieser Phase wurde das Ziel dieser Projektarbeit festgelegt. Vom Ziel
ausgehend wurden die dazu erforderlichen Projektphasen festgelegt.

\subsubsection{Erarbeitung theoretische Grundlagen}
\label{ssubsec:theoretical_background}

In dieser Phase ging es darum, sich in die Materie einzulesen und sich
diese anzueignen. Das Lesen und Bearbeiten von Publikationen führte zu
vielen neuen Denkanstössen und immer wieder zu neuen Recherchen. Die
Erkenntnisse dieser Phase flossen so stetig in die Dokumentation ein,
ergänzten und beeinflussten diese.

\subsubsection{Dokumentation}
\label{ssubsec:documentation}

Die vorliegende Arbeit entspricht der Dokumentation. Sie wurde während
der gesamten Projektarbeit stetig erweitert und diente zur Reflexion von
fertiggestellten Teilen.

\section{Technologien}
\label{sec:technologies}

\subsection{Tools und Software}
\label{subsec:tools_software}

\noindent\emph{Dokumentation und Prototyp}
\begin{description}
    \item[\LaTeX] Eine Makro-Sammlung für das \TeX-System. Wurde zur
        Erstellung dieser Dokumentation eingesetzt. Diese Dokumentation
        wurde mittels \LaTeX{} geschrieben.
    \item[Make] Build-Automations-Werkzeug, wurde zur Erstellung dieses Dokumentes eingesetzt.
    \item[CMake] Build-Automations-Werkzeug, wurde zur Erstellung des
        Prototypen eingesetzt.
    \item[zotero] Ein freies, quelloffenes Literaturverwaltungsprogramm
        zum Sammeln, Verwalten und Zitieren unterschiedlicher Online-
        und Offline-Quellen~\parencite{wikipedia_foundation_zotero_2015}.
    \item[VIM] Vi IMproved. Ein freier, quelloffener Texteditor zur
        Textbearbeitung. Wurde zum Verfassen der Dokumentation sowie zur
        Entwicklung des Prototypen eingesetzt.
    \item[LLVM] Low Level Virtual Machine. Eine Compiler-Architektur zum
        Kompilieren von Applikationen. Wurde zur Kompilation des
        Prototypen eingesetzt.
    \item[clang] Ein C-Sprachen-Frontend für LLVM.\ Wurde zur Kompilation
        des Prototypen eingesetzt.
\end{description}


\noindent\emph{Arbeitsorganisation}
\begin{description}
    \item[Git] Freie Software zur verteilten Versionsverwaltung, wurde
        für die Entwicklung dieser Dokumentation sowie des Prototypen verwendet. Die
        Projektarbeit findet sich
        unter~\href{https://www.github.com/sosterwalder/mte7101-project1}{GitHub}\footnote{
            \href{https://www.github.com/sosterwalder/mte7101-project1}{https://www.github.com/sosterwalder/mte7101-project1}
        }.
    \item[GitHub] Eine freie Hosting-Platform für Git mit Weboberfläche.
\end{description}

\subsection{Standards und Richtlinien}
\label{subsec:standards_guidelines}

\subsubsection{Programmcode}
\label{ssubsec:standards_guidelines:code}

Der Programmcode des Prototypen, welcher in C++ geschrieben wurde, folgt
den offiziellen Richtlinien für C++ von Google~\footnote{
    \href{https://google.github.io/styleguide/cppguide.html}{
        https://google.github.io/styleguide/cppguide.html
    }
}.

\subsubsection{Pseudecode}
\label{ssubsec:standards_guidelines:psuedocode}

Da der Autor dieser Arbeit bedingt durch seine tägliche Arbeit mit der
Programmiersprache \textit{Python} relativ bewandert ist, wird daher diese als
Sprache zur Beschreibung von Pseudocode verwendet.  Dabei wird aber kein
Augenmerk auf die formale Korrektheit, geschweige denn der Lauffähigkeit
des Pseudocodes gelegt.

\subsubsection{Projekt-Struktur}
\label{ssubsec:standards_guidelines:project_structure}

Um die Übersicht zu wahren und den Verwaltungsaufwand minimal zu halten,
wurde eine entsprechende Projekt-Struktur gewählt. Diese ist in
Auflistung~\ref{lst:project_structure} ersichtlich.

% Change caption
\begin{listing}
	\VerbatimInput[label=Projekt-Struktur,frame=single,numbers=left,firstline=23,lastline=39]{../README.md}
	\caption{Projekt-Struktur.}\label{lst:project_structure}
\end{listing}

% -*- coding: UTF-8 -*-
% vim: autoindent expandtab tabstop=4 sw=4 sts=4 filetype=tex
% chktex-file 27 - disable warning about missing include files

\chapter{Theoretischer Hintergrund}
\label{chap:theoretical_background}

% -*- coding: UTF-8 -*-
% vim: autoindent expandtab tabstop=4 sw=4 sts=4 filetype=tex
% chktex-file 27 - disable warning about missing include files

\section{Beleuchtungsmodelle}
\label{sec:illumination_models}

Sofern nicht anders vermerkt, basiert der folgende Abschnitt auf~\cite{whitted_improved_1980}[S. 343] sowie auf~\cite{hughes_computer_2013}.

Beleuchtungsmodelle beschreiben, wieviel Licht von einem sichtbaren Punkt einer Oberfläche zum Betrachter emitiert wird. In der Regel wird das Licht als Funktion in Abhängigkeit folgender Faktoren beschrieben:
\begin{itemize}
    \item Richtung der Lichtquelle \item Lichstärke
    \item Position des Betrachters
    \item Orientierung der Oberfläche
    \item Oberflächenbeschaffenheit
    \item Globale Umgebung
\end{itemize}

Es wird dabei zwischen lokalen und globalen Belechtungsmodellen unterschieden.

\subsection{Lokale Beleuchtungsmodelle}
\label{subsec:local_illumination_models}

Lokale Beleuchtungsmodelle aggregieren Daten von benachbarten, eben lokalen, Oberflächen. Diese Modelle sind in deren Umfang allerdings limitiert, da sie normalerweise nur Lichtquellen sowie die Orientierung einer Oberfläche einbeziehen. Sie ignorieren dabei aber die globale Umgebung, in welcher sich eine Oberfläche befindet.
Dies ist dadurch bedingt, dass die traditionell verwendeten Algorithmen zur Berechnung der Sichtbarkeit von Oberflächen, über keine globalen Daten verfügen.

Als Beispiel für ein lokales Beleuchtungsmodell dient das Phong-Beleuchtungsmodell, welches von Bui-Tong Phong entwickelt wurde.
Es beschreibt die reflektierte (Licht-) Intensität als Zusammensetzung aus der ambienten, der diffusen und der ideal spiegelnden Reflexion einer Oberfläche:
\begin{equation}
    I = I_{ambient} + I_{diffuse} + I_{specular}
\end{equation}
oder mathematisch ausgedrückt:
\begin{equation}
    I = I_a + k_d \displaystyle\sum_{j=1}^{ls} (\overrightarrow{N} \cdot \overrightarrow{L_j}) + k_s \displaystyle\sum_{j=1}^{ls} (\overrightarrow{N} \cdot \overrightarrow{L_j^`} )
\end{equation}
wobei gilt:
\begin{itemize}
    \item $I$:                      Die reflektierte (Licht-) Intensität
    \item $I_a$:                    Reflektion bedingt durch die Beleuchtung des Raumes
    \item $k_d$:                    Konstante für die diffuse Komponente des reflektierten Lichtes
    \item $\overrightarrow{N}$:     Einheitsnormale der Oberfläche
    \item $\overrightarrow{L_j}$:   Vektor in Richtung der $j$-ten Lichtquelle
    \item $k_s$:                    Koeffizient der spiegelenden Komponente
    \item $\overrightarrow{L_j^`}$: Vektor in der Hälfte zwischen dem Betrachter und der $j$-ten Lichtquelle
    \item $n$:                      Exponent, welcher von der Reflektion der Oberfläche abhängt
    \item $ls$:                     Anzahl Lichtquellen
\end{itemize}

\subsection{Globale Beleuchtungsmodelle}
\label{subsec:global_illumination_models}

Sofern nicht anders vermerkt, basiert der folgende Abschnitt auf~\cite{foley_computer_1996}[S. 775ff]

Globale Beleuchtungsmodelle beschreiben die reflektierte (Licht-) Intensität eines Punktes aufgrund direkter Lichteinstrahlung durch Lichtquellen sowie durch alles Licht, welches diesen Punkt nach Reflektion von bzw. Durchdringen der eigenen oder anderer Oberflächen erreicht.

Bei globalen Beleuchtungsmodellen unterscheidet man zwischen blickwinkelabhängigen Algorithmen, wie etwa Ray Tracing, und zwischen blickwinkelunabhängigen Algorithmen, wie etwa Photon Mapping.

Blickwinkelabhängige Algorithmen verwenden eine Diskretisierung~\todo{view plane} der sichtbaren Fläche um zu entscheiden, an welchen Punkten, in Blickrichtung des Betrachters, die Beleuchtungsberechnung durchgeführt werden soll. Blickwinkelunabhängige Algorithmen hingegen diskretisieren und verarbeiten die Umgebung um genügend Informationen für die Beleuchtungsberechnung zu haben. Dies erlaubt ihnen die Beleuchtungsberechnung an einem beliebigen Punkt aus einer beliebigen Blickrichtung.

Beide Arten von Algorithmen haben jedoch Vor- und Nachteile. So sind blickwinkelabhängige Algorithmen gut geeignet um Spiegelungen, basierend auf der Blickrichtung des Betrachtes, zu berechnen, eignen sich aber weniger um gleichbleibende diffuse Anteile über weiter Flächen eines Bildes zu berechnen. Bei blickwinkelabhängigen Algorithmen verhält es sich genau umgekehrt.

\subsubsection{Renderinggleichung}
\label{ssubsec:rendering_equation}

Die unter~\ref{subsec:global_illumination_models} genannten Verfahren versuchen auszudrücken, wie sich Licht von einem Punkt im Raum zu einem anderen bewegt. Dabei beschreiben sie die Intensität des Lichtes, ausgehend vom ersten Punkt zum zweiten Punkt. Zusätzlich wird die Intensität des Lichtes, ausgehend von allen anderen Punkten, welche den ersten Punkt erreichen, und zum zweiten Punkt emitiert werden, beschrieben.

James (Jim) Kajiya stellte 1986 die so genannte Renderinggleichung auf, welche genau dieses Verhalten beschreibt:
\begin{equation}
    I(x, x') = g(x, x')[\epsilon(x, x') + \int\limits_{s}\rho(x, x', x'')I(x', x'')dx'']
\end{equation}
wobei gilt:

\captionof{table}{Beschreibung der Komponenten der Renderinggleichung nach~\cite{kajiya_rendering_1986}[S. 143]}
\begin{tabular}{ l l }
    $ x', x' und x''   $: & Punkte in der Umgebung                                                                                                  \\
    $ I(x, x')         $: & Lichtintensität von Punkt $x'$ nach Punkt $x$                                                                           \\
    $ g(x, x')         $: & \parbox[t]{14cm}{Ein auf die Geometrie bezogener Term:                                                                  \\
                                 \hspace*{12mm} $0$:     \hspace*{6mm} $x$ und $x'$ verdecken sich                                                  \\
                                 \hspace*{12mm} $1/r^2$: \hspace*{1mm} $x$ und $x'$ sehen sich, wobei $r$ die Distanz zwischen $x$ und $x'$ ist}    \\
    $ \epsilon(x, x')  $: & Intensität des Lichtes, welches von $x'$ nach $x$ emitiert wird                                                         \\
    $ \rho(x, x', x'') $: & Intensität des Lichtes, welches von $x''$ durch die Oberfläche bei $x'$ nach $x$ gestreut wird                          \\
    $ \int\limits_{S}  $: & \parbox[t]{14cm}{Integral über die Vereinigung aller Flächen, daher $ S = \bigcup{S_{i}} $                              \\
                            Dies bedeutet, dass die Punkte $x$, $x'$ und $x''$ über alle Flächen aller Objekte der Szene ``streifen''.              \\
                            Wobei es sich bei $S_{0}$ um eine zusätzliche Fläche handelt, welche als Hintergrund verwendet wird.                    \\
                            $S_{0}$ ist dabei eine Hemisphäre, welche die gesamte Szene umspannt.}                                                  \\
\end{tabular}

\section{Ray Casting}
\label{sec:ray_casting}

Sofern nicht anders vermerkt, basiert der folgende Abschnitt
auf~\cite{hughes_computer_2013}[Kapitel 15, S. 387ff].\\
\\
Um ein Bild möglichst realistisch darzustellen muss berechnet werden, wieviel
Licht zu jedem Pixel der sichtbaren Bildfläche (also dem Betrachter)
transportiert wird. Da Photonen die Energie des Lichtes transportieren, muss
man also das physikalische Verhalten dieser simulieren. Es ist allerdings nicht
möglich \textit{alle} Photonen zu simulieren, da der Aufwand schlicht zu gross
wäre. Daher macht es Sinn nur einige Photonen (exemplarisch) zu betrachten und
dann eine Abschätzung des gesamten Lichtes vorzunehmen.\\
\\
Bei \textbf{Ray Casting} handlt es sich grundsätzlich um eine Strategie zur
Simulation, wieviel Licht anhand eines (Licht-) Strahles zu der sichtbaren
Bildfläche (also dem Betrachter) transportiert wird.

\begin{figure}[H]
    \centering \rotatebox{0}{\scalebox{0.3}[0.3]{\includegraphics{img/ray_tracing_01.png}}}
    \caption{Punkt $P$ auf einer Oberfläche eines Dreieckes, welcher für die Kamera bzw.\ den Betrachter sichtbar ist.
        Der Betrachter nimmt dabei das Licht, welches aus verschiedenen Richtungen $\omega_{i}$ kommt, über den Punkt $P$ in Richtung $\omega_{0}$ wahr.\label{fig:ray_casting:basics}\protect\footnotemark}
\end{figure}
\footnotetext{Darstellung von~\cite{hughes_computer_2013}[Kapitel 15, Seite 389, Abbildung 15.1]}

Wie in Abbildung~\ref{fig:ray_casting:basics} ersichtlich, gelangt Licht aus vielen Richtungen durch den Punkt $P$ zu dem Betrachter. Dies beinhaltet auch die Möglichkeit, dass
Licht nicht nur von einer Lichtquelle aus, sondern von vielen Lichtquellen aus via $P$ zum Betrachter gelangt. Weiter ist es möglich, dass Licht zuvor an anderen Punkten gestreut
und/oder gespiegelt und erst dann via $P$ zum Betrachter gelangte.\\
\\
Dies führt zu den folgenden Schlussfolgerungen:
\begin{itemize}
    \item Es müssen alle möglichen Richtungen, aus denen Licht kommen könnte,
        an Punkt $P$ untersucht werden.
    \item Da, bedingt durch technische Limitierungen, nur diskretes Abtasten
        möglich ist, müssen die Richtungen auf eine endliche Anzahl beschränkt
        werden, was zu Abtastfehlern führen kann.
\end{itemize}
Um die Abtastfehler zu minieren, können die Richtungen des Abtasten anhand der Lichtquellen priorisiert werden.

\newpage{}

Ein möglicher Algorithmus, wie solch ein Verfahren umgesetzt werden kann,
findet sich in~\ref{fig:ray_casting:high_level}.

\begin{python}[caption={Eine abstrakte Umsetzung des Ray
        Castings\protect\footnotemark}.,label={fig:ray_casting:high_level},captionpos=b]
def ray_cast():
    # "pixels" is a list of all pixels of the image plane
    for pixel in pixels:
        # Save all intersections for given pixel
        intersections = []

        # Returns the ray passing through the given
        # pixel from the eye
        ray = ray_at_pixel(pixel)

        # "scene_triangles" is a list of all triangles
        # coming from meshes contained in the scene to render
        for triangle in scene_triangles:
            p   = intersect(ray, triangle)
            sum = 0

            for light in incoming_lights_at_p:
                sum = sum + l.value
            end

            if is_smallest_intersection(p, intersections):
                pixel = sum
            intersections.append(p)
\end{python}
\footnotetext{Algorithmus in Pseudocode gemäss~\cite{hughes_computer_2013}[Kapitel 15, Seite 391, Auflistung 15.2]}

Das Verfahren wurde erstmals 1968 in der Publikation ``Some techniques for
shading machine renderings of solids'' von A. Appel vorgeschlagen und auch 1968
von der Matthematical Applications Group Inc.\ in ``3-D Simulated Graphics
Offered by Service Bureau'' erfolgreich umgesetzt.

\newpage{}

\section{Ray Tracing}
\label{sec:ray_tracing}

Bei dem heute als Ray Tracing bekannten Verfahren, handelt es sich um eine
verbesserte Version des unter~\ref{sec:ray_casting} genannten Ray Casting
Verfahrens. Dieses wurde im Juni 1980 durch Turner Whitted in der Publikation
``An Improved Illumination Model for Shaded Display'' verbessert.\\
\\
So schlägt Turner vor, dass die Berechnung der Sichtbarkeit (von Objekten)
nicht bei dem nähesten gefundenen Schnittpunkt abgebrochen wird, sondern dass
jedes Auftreffen eines (Licht-) Strahles mehr (Licht-) Strahlen durch
Transmission bzw. Reflektion sowie in Richtung jeder Lichtquelle gesendet
werden. Dieser Prozess wird so lange wiederholt, bis keiner der neu generierten
(Licht-) Strahlen mehr auf ein Objekt trifft~\cite{whitted_improved_1980}[S.
345].\\
Es handelt sich dabei also um ein rekursives Verfahren und wird daher
teilweise auch rekurisves Ray Tracing genannt.

% -*- coding: UTF-8 -*-
% vim: autoindent expandtab tabstop=4 sw=4 sts=4 filetype=tex
% chktex-file 27 - disable warning about missing include files

\section{Oberflächen}
\label{sec:surfaces}

Sofern nicht anders vermerkt, basiert der folgende Abschnitt
auf~\cite{division_introduction_1996}[S. 1 ff].\\
\\
Um in Computergrafiken überhaupt etwas darstellen zu können, muss erst einmal
definiert werden, was dargestellt werden soll. Häufig orientiert sich die
Computergrafik dabei an der realen Welt.  In der realen Welt haben Oberlächen
von Objekten häufig keine starken Übergänge (Kanten) sondern sind eher von
glatter Natur~\cite{foley_computer_1996}[S. 471].\\
\\
Die Darstellung von Kurven und Oberflächen führt zu zwei Fällen: Modellierung
von bestehenden Objekten und Modellierung von Grund auf.\\
\\
Zur Modellierung von Oberflächen werden hauptsächlich zwei Techniken verwendet:
Parametrische Modellierung und implizite Modellierung.\\
\\
Bei der parametrsichen Darstellung wird eine Oberfläche überlicherweise als
eine Menge von Punkten definiert, so zum Beispiel:

\begin{gather}\label{eq:surface_parametric}
    \bm{p}(s, t) = (x(s, t), y(s, t), z(s, t))
\end{gather}

Bei der impliziten Darstellung wird eine Oberfläche überlicherweise als Kontur
einer Funktion mit Wert 0 definiert, so zum Beispiel:

\begin{gather}\label{eq:surface_implicit}
    f(\bm{p}) = f(x, y, z) = 0
\end{gather}

Die parametrische Darstellung bringt Vorteile wie die Unabhängigkeit von einem
Koordinatensystem oder eine effiziente Berechnung von Punkten auf einer
Oberfläche. Die implizite Darstellung erlaubt hingegen eine grössere
Einflussnahme aus mathematischer Sicht und ist daher sehr nützlich für
Operationen wie Biegung, Vermischung, Schnitte (Intersektion) oder Bool'sche
Operationen.

\todo[inline]{Meh. Not sure if intro is good enough.}

\subsection{Implizite Oberflächen}
\label{subsec:implicit_surfaces}

Wie in Gleichung~\ref{eq:surface_parametric} beschrieben, ist eine implizite
Oberfläche gemäss~\cite{hart_ray_1993}[S. 1] als Funktion $ f(\bm{x}) =
\mathbb{R}^{3} \to \mathbb{R} $ definiert.  Es wird also jedem Punkt einer
Menge $ \bm{p} \in \mathbb{R}^{3} $ ein skalarer Wert $ s \in \mathbb{R} $
zugewiesen. Dabei besteht die Oberfläche aus der Punktemenge $ \bm{x} \equiv
(x, y, z) \in \mathbb{R}^{3} $.\\
\\
Angenommen $ \bm{A} $ ist ein geschlossener Festkörper, welcher durch die
Funktion $f$ beschrieben wird, dann kann gemäss~\cite{hart_ray_1993} Folgendes
angenommen werden:

\begin{gather} \label{eq:surface_implicit_condition}
    x \in \overset{\circ}{\bm{A}} \Leftrightarrow f(\bm{x}) < 0 \\
    x \in \partial \bm{A}         \Leftrightarrow f(\bm{x}) = 0 \\
    x \in \mathbb{R}^{3} - \bm{A} \Leftrightarrow f(\bm{x}) > 0
\end{gather}

Dies bedeutet, dass die implizite Funktion $ f(\bm{x}) $
\begin{itemize}
    \item negativ ist, wenn sich ein Punkt $\bm{x}$ innerhalb von $\bm{A}$
        befindet
    \item 0 ist, wenn sich ein Punkt $\bm{x}$ auf der Oberfläche von $\bm{A}$
        befindet
    \item Positiv ist, wenn sich ein Punkt $\bm{x}$ ausserhalb von $\bm{A}$
        befindet
\end{itemize}

Dies gilt, da es sich bei $ \bm{x} $ um eine Punktemenge mit topologischer
Struktur handelt.\\
\\
Gemäss~\cite{division_introduction_1996} finden hauptsächlich drei Methoden
Anwendung zur Beschreibung impliziter Oberflächen: algebraische Oberflächen,
Blobby-Objekte sowie die funktionale Repräsentation.~\cite{hart_sphere_1994}
gibt jedoch an, dass die gebräuchlichste Form von impliziten Oberflächen die
algebraischen Oberflächen sind. Diese werden implizit durch polynomiale
Funktionen definiert.\\

\subsubsection{Algebraische und geometrische implizite Oberflächen}
\label{ssubsec:implicit_surfaces_algebraic_geometric}

Ein Beispiel für eine algebraische Oberfläche ist die Beschreibung der
Einheitskugel anhand einer impliziten algebraischen Gleichung zweiten Grades:

\begin{gather} \label{eq:surface_immplicit_algebraic}
    x^{2} + y^{2} + z^{2} - 1 = 0
\end{gather}

Wobei es sich bei dem letzten Parameter um den Radius $r$ handelt, welcher ---
bedingt durch die Einheitskugel --- den Wert 1 hat.

Wie~\cite{division_introduction_1996} schreibt, handelt es sich bei impliziten
Oberflächen, welche durch eine polynomiale Funktion zweiten Grades beschrieben
werden, um quadrische implizite Oberflächen.

\cite{hart_sphere_1994} gibt weiter an, dass --- unter Nuztung einer Metrik ---
die Einheitskugel durch die implizite Gleichung

\begin{gather} \label{eq:surface_immplicit_geometric}
    \|\bm{x}\| - 1 = 0
\end{gather}

beschrieben werden kann, was unter Anwendung der allgemeinen
Form~\ref{eq:surface_implicit} einer impliziten
Gleichung~\ref{eq:surface_implicit} zu folgender Gleichung führt:

\begin{gather}
    f(\bm{x}) = \|\bm{x}\| - 1
\end{gather}

Dabei ist $\|\bm{x}\|$ als euklidische Metrik definiert und entspricht $\sqrt{x^{2} + y^{2} + z^{2}}$.\\
\\
Die Gleichung~\ref{eq:surface_immplicit_algebraic} gibt die algebraische
Distanz zurück, Gleichung~\ref{eq:surface_immplicit_geometric} gibt die
geometrische Distanz zurück.\\

\subsubsection{Distanzfunktionen}
\label{ssubsec:distance_functions}

Gemäss~\cite{hart_sphere_1994} wird die geometrische Darstellung von
quadrischen Oberflächen bevorzugt, da deren Parameter unabhängig von
Koordinaten sind, sie robuster und intuitiver sind. Es handelt sich dabei um
eine \textbf{Distanzfunktion}.\\
\\
Wie anfangs erwähnt, definiert~\cite{hart_sphere_1994} die allgemeine Form zur
Beschreibung bzw. Darstellung von impliziter Oberflächen als Zuweisung von
Punkten zu einem skalaren Wert: $ f : \mathbb{R}^{n} \to \mathbb{R} $.\\
\\
Unter Anwendung der unter~\ref{eq:surface_implicit_condition} definierten
Bedingungen kann geschlossen werden, dass eine Menge von Punkten $A$ existiert,
welche Teil von $\mathbb{R}^{n}$, also $A \subset \mathbb{R}^{n}$ ist. Dies heisst, dass alle Punkte in $A$ die folgende Bedingung erfüllen:

\begin{gather}
    A = \{ x : f(x) \leq 0 \}
\end{gather}

\cite{hart_sphere_1994} liefert zwei Definition, welche der Beschreibung von Distanzfunktionen dienen:

\theoremstyle{definition}
\begin{definition}{\label{theo:point_to_set_distance}
    \textit{Point-to-set distance}}\\
    Die Distanz eines Punktes zu einer Menge von Punkten definiert die Distanz
    eines Punktes $ \bm{x} \in \mathbb{R}^{3} $ zu einer Menge von Punkten $A
    \subset \mathbb{R}^{3}$ als Distanz von $\bm{x}$ zum nähesten Punkt in $A$:

    \begin{gather}
        d(\bm{x}, \bm{A}) = \min_{\substack{\bm{y} \in \bm{A}}}(\|\bm{x} - \bm{y}\|)
    \end{gather}
\end{definition}

\theoremstyle{definition}
\begin{definition}{\label{theo:signed_distnace_bound}
    \textit{Signed distance bound}}\\ 
    Eine Funktion $ f : \mathbb{R}^{3} \to \mathbb{R} $ ist eine Obergrenze
    ihrer impliziten Oberfläche $ f^{-1}(0)$, wenn gilt:

    \begin{gather}\label{eq:signed_distnace_bound}
        |f(\bm{x})| \leq d(\bm{x}, f^{-1}(0))
    \end{gather}
\end{definition}

Wenn die Gleichung~\ref{eq:signed_distnace_bound} für eine Funktion $f$ gilt,
dann ist $f$ eine \textit{vorzeichenabhängige Distanzfunktion (signed distance
    function)}.

\subsubsection{Distanzfelder (distance fields)}
\label{ssubsec:distance_fields}

\todo[inline]{Introduce distance fields.}

\section{Darstellung von impliziten Oberflächen}
\label{sec:rendering_implicit_surfaces}

Wie~\cite{hart_sphere_1994}[S. 1] angibt, existieren verschiedene Möglichkeiten
zur Darstellung (zum Rendering) von impliziten Oberflächen. So wandeln
indirekte Methoden implizite Oberflächen in Polygonmodelle um, was die Nuztung
bestehender Techniken und Hardware zur Darstellung von polygonalen Modellen
erlaubt. Obwohl die Umwandlung der impliziten Oberflächen mit gängigen Systemen
zur Darstellung problemlos dargestellt werden kann, ist die Umwandlung jedoch nicht
in jedem Fall gegeben und kann zu nicht zusammenhängenden Flächen oder einer
Verminderung des Detailgrades führen.\\
\\
Eine andere Methode zur Darstellung von impliziten Oberflächen ist das
unter~\ref{sec:ray_tracing} vorgestellte Ray Tracing Verfahren.

Ein (Licht-) Strahl wird dabei parametrisch als

\begin{gather}\label{eq:ray_param}
    r(t) = r_{0} + t * r_{d}
\end{gather}

beschrieben. Der Strahl startet dabei bei Punkt $r_{0}$ in Richtung des
Einheitsvektors $r_{d}$, wobei $t$ die zurückgelegte Distanz des Strahles ist.
$r(t)$ räpresentiert also den Punkt in Raum, welchen der Strahl nach dem
Zurücklegen der Distanz $t$ --- ausgehend von seinem Ursprung $r_{0}$ ---
erreicht.\\
\\
Um nun die Schnittpunkte eines Strahles mit einer impliziten Oberfläche zu finden, wird die Gleichung des Lichtstrahles $r$ (\ref{eq:ray_param}) in die Funktion einer impliziten Oberfläche $f$ (\ref{eq:surface_implicit}) eingesetzt. Wobei $r : \mathbb{R} \to \mathbb{R}^{3}$ und $f : \mathbb{R}^{3} \to \mathbb{R}$. Dies ergibt die zusammengesetzte Funktion $F = f \circ r$ wobei $F : \mathbb{R} \to \mathbb{R}$.\\
\\
Die Lösungen dieser Gleichung sind alle Distanzen $t$, welche ein gegebener Strahl zurücklegt und welche die folgende Bedingung erfüllen:

\begin{gather}\label{eq:ray_param_cond}
    F(t) = f \circ r = f(r(t)) = 0
\end{gather}

Um die Gleichung~\ref{eq:ray_param_cond} zu lösen, können numerische Verfahren
zur Nullstellensuche angewendet werden, wobei die Verfahren vom Typ der
Funktion $F(t)$ abhängig sind. Bei polynomialen Funktionen bis zum vierten Grad
existieren analytische Lösungen, für eine beliebige Funktion muss jedoch ein
generisches, robustes Verfahren zur Nullstellensuche verwendet werden. Dies
bedingt jedoch meist, dass mehr Informationen über die Funktion zur Verfügung
stehen, was beispielsweise durch Ableiten dieser gelöst werden kann.\\
\\
Die erwähnten Verfahren zur Nullstellensuche haben jedoch häufig den Nachteil,
dass sie mehrere Schnittpunkte eines Strahles mit einer impliziten Oberfläche
liefern. Um diese Problematik zu umgehen, wird nur der kleinste Wert von $t$
berücksichtigt. Die Ray Marching und Sphere Tracing Algorithmen gehen hier
sogar noch einen Schritt weiter, in dem sie nur die erste Nullstelle der
Gleichung~\ref{eq:ray_param_cond} betrachten.

\subsection{Ray Marching}
\label{subsec:ray_marching}

~\cite{perlin_hypertexture_1989} schlagen eine Abtastung des Strahles mit fixen
Abständen $\Delta \mu$ vor:

\begin{gather}
    x = x_{\mu_{0}} + k * \Delta x_{\mu}
\end{gather}

wobei $k = 0,1,2,\dots$ und $\mu_{0} + k \Delta \mu \leq \mu_{1}$.\\
\\
Auf die parametrische Darstellung eines (Licht-) Strahles,
Gleichung~\ref{eq:ray_param_cond}, angwendet:

\begin{gather}
    r(k) = r_{0} + \Delta t * k * r_{d}
\end{gather}

wobei $\Delta t$ die Grösse der Abstände und $k = 0,1,2,\dots$ die Nummer der
Schritte darstellt. Wie~\cite{hart_ray_1989} schreiben, bildet das Abtasten des
(Licht-) Strahles mit fixen Abständen die Basis für gewisse Verfahren des
volumetrischen Renderings.\\
\\
Ein möglicher Algorithmus, wie solch ein Verfahren umgesetzt werden kann,
findet sich in~\ref{fig:ray_marching}.\\
\\
\begin{python}[caption={Eine abstrakte Umsetzung des Ray
        Marchings\protect\footnotemark}.,label={fig:ray_marching},captionpos=b]
def ray_march():
    step         = 0
    intersection = 0
    max_steps    = 10

    while step < max_steps:
        intersection = test_intersection(k)

        if intersection <= 0:
            # An intersection has happened
            #   intersection <  0: ray is inside surface
            #   intersection == 0: ray is excatly on surface
            return ray_travel_distance(step)

        step = step + 1

    # When w reach this step, after max_steps, no intersection
    # has happened, so distance is 0
    return 0
\end{python}
\footnotetext{Algorithmus in Pseudocode
    gemäss~\cite{perlin_hypertexture_1989}[S. 259, Abschnitt 3.1]}

Dabei ist jedoch zu beachten, dass der Abstand zur Abtastung eines Strahles
$\Delta t$ so gering als möglich sein sollte um eine Punktemeng bzw.\ ein
Objekt --- definiert durch implizite Oberflächen --- $A$ möglichst gut
abschätzen zu können. Ist der gewählte Abstand zu gross gewählt, so findet ggf.
eine Abtastung weit im Inneren des Objektes statt und somit geht Präzision
verloren.  Es ist auch möglich dass der erste eigentliche Punkt gar nicht
abgetastet wird und erst der zweite abgetastete Punkt das Objekt ``erkennt''.
Die Grafik~\todo{insert reference to image here} veranschaulicht diese
Problematiken.\\
\\
\cite{hart_sphere_1994} weist darauf hin, dass Ray Marching durch den möglichst
geringen Abstand zwischen den Abtastungen entsprechend langsam und paralleles
Abtasten praktisch unumgänglich ist. In der von~\cite{hart_sphere_1994}
vorgestellten Technik des Sphere Tracings ist der Abstand zwischen den
Abtastungen nicht konstant sondern variiert in Abhängigkeit der Geometrie.

\subsection{Sphere Tracing}
\label{subsec:sphere_tracing}



% -*- coding: UTF-8 -*-
% vim: autoindent expandtab tabstop=4 sw=4 sts=4 filetype=tex
% chktex-file 27 - disable warning about missing include files

\chapter{Diskussion und Fazit}
\label{chap:discussion_and_conclusion}

\section{Diskussion}
\label{sec:discussion}

\section{Erweiterungsmöglichkeiten}
\label{sec:further_work}

\section{Fazit}
\label{sec:fazit}


%---------------------------------------------------------------------------

% Glossary
%---------------------------------------------------------------------------
\cleardoublepage{}
\phantomsection{}
\addcontentsline{toc}{chapter}{Glossar}
\renewcommand{\glossaryname}{Glossar}
\glsaddall{}
\printglossaries{}
%---------------------------------------------------------------------------

% Bibliography
%---------------------------------------------------------------------------
%\cleardoublepage
\phantomsection{}
\addcontentsline{toc}{chapter}{Literaturverzeichnis}
\bibliographystyle{unsrtnat}
\bibliography{inc/static/bibliography}{}
%---------------------------------------------------------------------------

% Listings
%---------------------------------------------------------------------------
%\cleardoublepage
\phantomsection{}
\addcontentsline{toc}{chapter}{Abbildungsverzeichnis}
\listoffigures
%\cleardoublepage
\phantomsection{}
\addcontentsline{toc}{chapter}{Tabellenverzeichnis}
\listoftables
%\cleardoublepage
\phantomsection{}
\addcontentsline{toc}{chapter}{Auflistungsverzeichnis}
\lstlistoflistings{}
%---------------------------------------------------------------------------

% Index
%---------------------------------------------------------------------------
%\cleardoublepage
%\phantomsection{}
%\addcontentsline{toc}{chapter}{Stichwortverzeichnis}
%\renewcommand{\indexname}{Stichwortverzeichnis}
%\printindex
%---------------------------------------------------------------------------

% Attachment:
%---------------------------------------------------------------------------
%\appendix
\settocdepth{section}
% -*- coding: UTF-8 -*-
% vim: autoindent expandtab tabstop=4 sw=4 sts=4 filetype=tex
% chktex-file 27

% In den Anhang fügen Sie ein:
%  * Details des Projektpans, falls vorhanden
%  * Resultate und Zwischenresultate in Funktion der Projektiterationen
%  * Pflichtenheft / Anforderungsspezifikation (Stand Ende dritter Woche)
%  * Angaben zum Projektrepository
%  * Sitzungsprotokolle, falls vorhanden
%  * Weiterführende Erläuterungen zu den verwendeten Technologien, falls nötig
%  * Benutzerhandbuch, falls vorhanden und sinnvoll, es hier aufzulisten
%  * Installations- und Betriebsdokument, falls vorhanden und sinnvoll, es hier aufzulisten
% Unterlassen Sie das Anfügen von Listings.

\appendix 
\begin{titlepage}
    \clearpage
    \vspace*{\fill}
    \begin{center}
        \begin{minipage}{.6\textwidth}
            \fontsize{26pt}{28pt}\selectfont
            \chapter*{Anhang}\label{chap:attachment}
            \addcontentsline{toc}{chapter}{Anhang}
        \end{minipage}
    \end{center}
    \vfill % equivalent to \vspace{\fill}
    \clearpage
\end{titlepage}

\newpage
 % -*- coding: UTF-8 -*-
% vim: autoindent expandtab tabstop=4 sw=4 sts=4 filetype=tex

\chapter{Meeting minutes}
\label{chap:10_meeting_minutes}

\VerbatimInput[label=20150921]{inc/static/attachment/minutes/20150921}
\newpage
\VerbatimInput[label=20151004]{inc/static/attachment/minutes/20151004}
\newpage
\VerbatimInput[label=20151011]{inc/static/attachment/minutes/20151011}
\newpage
\VerbatimInput[label=20151011]{inc/static/attachment/minutes/20151018}
\newpage
\VerbatimInput[label=20151011]{inc/static/attachment/minutes/20151025}
\newpage
\VerbatimInput[label=20151011]{inc/static/attachment/minutes/20151102}
\newpage
\VerbatimInput[label=20151011]{inc/static/attachment/minutes/20151115}
\newpage
\VerbatimInput[label=20151011]{inc/static/attachment/minutes/20151206}



% \includepdfset{pagecommand={\thispagestyle{headings}}}
% \includepdf[pages=-, addtotoc={1,chapter,0,Anforderungsdokument,chap:anf},scale=0.95]{anhang/anforderungen.pdf}
% \newpage
% \includepdf[pages=-, addtotoc={1,chapter,0,Tutorial Wissensmodellierung,chap:tutorial},scale=0.95]{anhang/Tutorial.pdf}
% \newpage
% \input{anhang/schnipsel}
% \newpage
% \input{anhang/modellierung}
% \newpage
% \input{anhang/installationshandbuch}
% \newpage
% \includepdf[pages=-, addtotoc={1,chapter,0,Arbeitsjournal ,chap:arbeitsjournal},scale=0.95]{anhang/Journal.pdf}
% \newpage

%---------------------------------------------------------------------------

\end{document}
